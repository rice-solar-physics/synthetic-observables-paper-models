%%%%%%%%%%%%%%%%%%%%%%%%%%%%%%%%%%%%%%%%%%%%%%%%%%%%%%%%%%%%%%%%%%%%%%%%%%%%%%%
%                               Discussion                                    %
%%%%%%%%%%%%%%%%%%%%%%%%%%%%%%%%%%%%%%%%%%%%%%%%%%%%%%%%%%%%%%%%%%%%%%%%%%%%%%%

\section{Discussion}\label{discussion}

For all of our heating models, we find negative time lags in at least one of the three channel pairs as shown in \autoref{fig:timelag-maps}. Negative time lags can be used to disambiguate the temperature sensitivity of the AIA passbands and can be produced in one of two ways: high-frequency heating in which the time lag is dominated by many frequent reheatings or a channel pair in which one channel is bimodal. While intensity in the 131 \AA{} channel can correspond to either $<0.4$ MK or $>10$ MK plasma (see \autoref{fig:aia-response}), negative time lags in the 211-131 \AA{} channel pair provide a possible signature of $\ge10$ MK plasma because a negative time lag implies the plasma is cooling from 131 \AA{} to 211 \AA{}. This also holds for the 171-131 and 193-131 \AA{} channel pairs as well while the 94-131 and 335-131 \AA{} channels are more ambiguous due to the first channel in the pair being bimodal as well. As noted in \autoref{timelag_maps}, the high-, intermediate-, and low-frequency maps for the 211-131 \AA{} channel pair all show coherent negative time lags in the core. Because these strands are rooted in areas of strong field, enough energy is made available by the field (see \autoref{heating}) to heat them well into (and likely above) the hot component of the 131 \AA{} passband. Since these strands are relatively short, the density increases rapidly enough for this hot plasma to be visible before it is washed out by thermal conduction.

Plasma undergoing pure cooling by radiation and thermal conduction produces a predictable and well-understood time-lag signature. However, complicated heating scenarios and LOS geometries are likely to make it more difficult to interpret observed time-lag signatures. Consider the case of a single cooling strand such that the maximum allowed time lag for a given channel pair $AB$ is the amount of time it takes to cool from $T_A$ to $T_B$ by thermal conduction and radiation. We may regard the ``cooling'' case in \autoref{fig:timelag-histograms} as the baseline time-lag distribution given that all strands were heated only once at $t=0$ s. Because the time lag is primarily determined by the cooling phase of the strand, the time lag becomes primarily a function of the loop length $L$ since $\tau_\textup{cool}\propto L$. Two effects are likely to increase the decoherence of the baseline time-lag distribution: multiple structures evolving out-of-phase along a given LOS (the ``random'' heating scenario) and multiple reheatings before the end of the cooling and draining cycle on a given strand. We note that multiple polluting structures along the LOS seem to primarily add negative time lags to the distribution (the ``random'' case) while increasing the frequency of events on a given strand extends the distribution in the positive direction. The latter effect also produces more negative time lags. Since steady heating can be thought of as nanoflare heating in the high-frequency limit ($\langle\twait\rangle\to0$), we expect the distribution of time lags to approach a uniform distribution as the heating frequency increases \added{because variations in the intensity will be increasingly dominated by noise} \citep{viall_signatures_2016}. \deleted{who found that} \added{In other words,} steadily-heated loops have no preferred time lag.

While our model for the energy deposition (see \autoref{heating}) does not assume any specific physical heating mechanism, the parameterization of the heating frequency in \autoref{eq:heating_types} has an interesting implication in the context of the \citet{parker_nanoflares_1988} nanoflare model. Rearranging \autoref{eq:heating_types} and recalling that $\tau_\textup{cool}\propto L$ gives $\langle\twait\rangle\propto L$, i.e. longer strands have longer absolute waiting times between heating events. Given that longer field lines tend to be rooted in regions of weaker magnetic field, this further implies that, where the field is stronger, energy is more quickly dissipated. According to \citet{parker_nanoflares_1988}, this dissipation is due to small-scale reconnection of flux tubes that are continually stressed by the convective motion of the underlying photosphere. Thus, in this context, our heating model implies that the reconnection and the underlying driver are more efficient in areas where the field is strongest.

Though we have not addressed it here, another possible mechanism for producing time-varying intensity in \AR s is thermal non-equilibrium (TNE) wherein condensation cycles driven by highly-stratified, but steady footpoint heating lead to long-period intensity pulsations \citep{kuin_thermal_1982}. Though originally used to explain coronal rain \citep{antolin_coronal_2010,antolin_multithermal_2015,auchere_coronal_2018} and prominences \citep{antiochos_model_1991}, several workers \citep{mok_three-dimensional_2016,winebarger_investigation_2016,froment_long-period_2017,winebarger_identifying_2018,froment_occurrence_2018} have recently claimed that TNE can produce time-lag signatures similar to those of impulsive heating models, suggesting that observed time lags may be consistent with both impulsive and steady heating. However, it is not yet clear whether TNE is consistent with observed signatures of very hot (8-10 MK) plasma. Detailed comparisons between TNE and nanoflare simulations and observations are desperately needed.
