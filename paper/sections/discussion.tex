%%%%%%%%%%%%%%%%%%%%%%%%%%%%%%%%%%%%%%%%%%%%%%%%%%%%%%%%%%%%%%%%%%%%%%%%%%%%%%%
%                               Discussion                                    %
%%%%%%%%%%%%%%%%%%%%%%%%%%%%%%%%%%%%%%%%%%%%%%%%%%%%%%%%%%%%%%%%%%%%%%%%%%%%%%%

\section{Discussion}\label{discussion}

The results of our numerical experiment show that signatures of the heating frequency persist in the emission measure slope and timelag. In high-frequency heating, the plasma is kept within a narrow temperature and density range because each strand is reheated often enough that it is not allowed to cool and drain significantly before being reenergized. This leads to relatively steep emission measure slopes (left panel of \autoref{fig:em-slope-maps}) which are broadly distributed (\autoref{fig:em-slope-histograms}) and spatially-decoherent timelags (\autoref{fig:timelag-maps}). In contrast, an infrequently heated strand is allowed to fully cool and drain before the next heating event and samples a wide range of temperatures and densities as illustrated in \autoref{fig:hydro-profiles}. This leads to emission measure slopes which are narrowly distributed with a mean between $a=2$ and $a=2.5$ and timelags which are relatively spatially-coherent. The maximum value of the cross-correlation, $\max\mathcal{C}_{AB}$, also increases with decreasing heating frequency (see \autoref{fig:correlation-maps}).

As the heating frequency decreases, the timelag maps in \autoref{fig:timelag-maps} become increasingly spatially coherent. Consistent with \citet{viall_signatures_2016}, who showed that steadily-heated loops have no preferred timelag, we find that the timelags in the high frequency case are determined primarily by the heating phase and are thus broadly distributed. However, if a strand is allowed to cool uninterrupted, the timelag will be determined primarily by the cooling phase of the strand evolution such that the spatial distribution of timelags becomes primarily a function of the loop length $L$ since $\tau_{cool}\propto L$.

While the timelag for a strand only cooling by radiation and thermal conduction is relatively easy to predict, multiple factors are likely to make it difficult to interpret this diagnostic if the heating scenario is more complicated. Consider the case of a single cooling strand such that the maximum allowed timelag for given channel pair $AB$ is the amount of time it takes to cool from $T_A$ to $T_B$ by thermal conduction and radiation. We may regard the ``cooling'' case in \autoref{fig:timelag-histograms} as the baseline timelag distribution given that all strands were heated only once at $t=0$ s. Two effects are likely to increase the decoherence of the baseline timelag distribution: multiple structures evolving out-of-phase along a given LOS (the ``random'' heating scenario) and multiple reheatings on a given strand before the end of the cooling cycle. We note that multiple polluting structures along the LOS seem to primarily add negative timelags to the distribution (the ``random'' case) while increasing the frequency of events on a given strand extends the distribution in the positive direction. The latter effect also produces more negative timelags. Since steady heating can be thought of as nanoflare heating in the high-frequency limit, we expect the distribution of timelags to approach a uniform distribution as the heating frequency increases. This is again consistent with the results of \citet{viall_signatures_2016} who found that steadily-heated loops have no preferred timelag.

Lastly, we note the importance of negative timelags in disambiguating the temperature sensitivity of the AIA passbands. Negative timelags can be produced in one of two ways: high-frequency heating in which the timelag is dominated by many frequent reheatings or a channel pair in which one channel is bimodal. While intensity in the 131 \AA{} channel can correspond to either $<0.4$ MK or $>10$ MK plasma (see \autoref{fig:aia-response}), negative timelags in the 211-131 \AA{} channel pair provide a possible signature of $\ge10$ MK plasma because a negative timelag implies the plasma is cooling from 131 \AA{} to 211 \AA{}. This also holds for the 171-131 and 193-131 \AA{} channel pairs as well while the 94-131 and 335-131 \AA{} channels are more ambiguous due to the first channel in the pair being bimodal as well. As noted in \autoref{timelag_maps}, the high-, intermediate-, and low-frequency maps for the 211-131 \AA{} channel pair all show coherent negative time lags in the core. Because these strands are rooted in areas of strong field, enough energy is made available by the field (see discussion in \autoref{heating}) to heat them well into (and likely above) the hot component of the 131 \AA{} passband. Since these strands are relatively short, the density increases rapidly enough for this hot plasma to be visible before it is washed out by thermal conduction.
