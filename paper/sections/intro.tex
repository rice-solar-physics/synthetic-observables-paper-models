%%%%%%%%%%%%%%%%%%%%%%%%%%%%%%%%%%%%%%%%%%%%%%%%%%%%%%%%%%%%%%%%%%%%%%%%%%%%%%%
%                                   Introduction                              %
%%%%%%%%%%%%%%%%%%%%%%%%%%%%%%%%%%%%%%%%%%%%%%%%%%%%%%%%%%%%%%%%%%%%%%%%%%%%%%%
\section{Introduction}\label{introduction}

Nanoflares have long been used to explain the observed million-degree temperatures in the non-flaring solar corona. Though originally pertaining to energetic bursts of order $10^{24}$ erg resulting from small-scale reconnection \citep{parker_nanoflares_1988}, the term \textit{nanoflare} is now synonymous with any impulsive energy release and is not specific to any particular physical mechanism \citep{klimchuk_key_2015}. Due to their faint, transient nature, direct observations of nanoflares are made difficult by several factors, including inadequate spectral coverage of instruments, the efficiency of thermal conduction, and non-equilibrium ionization \citep{cargill_implications_1994,winebarger_defining_2012,barnes_inference_2016}. However, recent observations of ``very hot'' 8-10 MK plasma, the so-called ``smoking gun'' of nanoflares, have provided compelling evidence for their existence \citep[e.g.][]{brosius_pervasive_2014,caspi_new_2015,parenti_spectroscopy_2017,ishikawa_detection_2017}.

Critical to understanding the underlying heating mechanism is knowing whether the corona in non-flaring active regions is heated \textit{steadily} or \textit{impulsively}, or, more precisely, at what frequency do nanflares repeat on a given magnetic strand. In the case of low-frequency nanoflares, the time between consecutive events on a strand is long relative to its characteristic cooling time, giving the strand time to fully cool and drain before it is reenergized. In the high-frequency scenario, the time between events is short relative to the cooling time such that the strand is not allowed to fully cool before being heated again. Steady heating may be regarded as nanoflare heating in the very high-frequency limit. 

Before proceeding, we note that a magnetic \textit{strand}, the fundamental unit of the low-$\beta$ corona, is a flux tube oriented parallel to the magnetic field that is isothermal in the direction perpendicular to magnetic field. We make the distinction that a \textit{coronal loop} is an observationally-defined feature representing a field-aligned intensity enhancement relative to the surrounding diffuse emission such that a single coronal loop may be composed of many thermally-isolated strands. Furthermore, we define the \AR{} \textit{core} as the area near the center of the \AR{} whose X-ray and EUV emission is dominated by closed loops with both footpoints rooted in the photosphere.

In lieu of a direct observable signature of nanoflare heating, two parameters in particular have been used to diagnose the heating frequency in \AR{} cores: the emission measure slope and the time lag. These diagnostics provide \textit{indirect} signatures of the energy deposition via observations of the plasma cooling by thermal conduction, enthalpy, and radiation. We now discuss each of these observables in detail.

The emission measure distribution, $\mathrm{EM}(T)=\int\mathrm{d}h\,n_e^2$, where $n_e$ is the electron density and the integration is taken along the line of sight, is a useful diagnostic for parameterizing the frequency of energy deposition. Many observational and theoretical studies have suggested that the ``cool'' portion of the \dem{} (i.e. leftward of the peak, $10^{5.5}\lesssim T\lesssim10^{6.5}$ K), can be described by $\mathrm{EM}(T)\sim T^a$ \citep{jordan_structure_1976,cargill_implications_1994,cargill_nanoflare_2004}. The so-called \textit{emission measure slope}, $a$, is an important diagnostic for assessing how often a single strand may be reheated and has been used by several researchers to interpret \AR{} core observations in terms of both high- and low-frequency heating \citep[see Table 3 of][and references therein]{bradshaw_diagnosing_2012}. The ``cool'' emission measure slope typically falls in the range $2<a<5$, with shallower slopes indicative of low-frequency heating and steeper slopes associated with high-frequency heating. Many observational studies of active region cores have used the emission measure slope to make conclusions about the heating frequency \citep[e.g.][]{tripathi_emission_2011,warren_constraints_2011,winebarger_using_2011,schmelz_cold_2012,warren_systematic_2012,del_zanna_evolution_2015}.

To better understand observable properties of nanoflare heating, several researchers have used hydrodynamic models of coronal loops to examine how the emission measure slope varies with heating frequency \citep{mulu-moore_can_2011,bradshaw_diagnosing_2012,reep_diagnosing_2013}. Most recently, \citet{cargill_active_2014} found that varying the time between consecutive heating events from 250 s (high-frequency heating) to 5000 s (low-frequency heating) could account for the wide observed distribution of emission measure slopes, with higher values of $a$ corresponding to higher heating frequency due to the \dem{} distribution becoming increasingly isothermal \citep[see also][]{barnes_inference_2016-1}.

In addition to the emission measure slope, the time-lag analysis of \citet{viall_evidence_2012} has also been used by several workers to understand the frequency of energy release in \AR{} cores. The \textit{time lag} is the temporal delay which maximizes the cross-correlation between two timeseries, and, qualitatively, can be thought of as the amount of time which one signal must be shifted relative to another in order to achieve the best ``match'' between the two signals. As the plasma cools through the six EUV channels of the Atmospheric Imaging Assembly instrument \citep[AIA,][]{lemen_atmospheric_2012} onboard the Solar Dynamics Observatory spacecraft \citep[SDO,][]{pesnell_solar_2012}, we expect to see the intensity peak in successively cooler passbands of AIA according to the sensitivity of each channel in temperature space \citep{viall_patterns_2011}. Computing the time lag between light curves in different channels provides a proxy for the cooling time between channels and insight into the thermal evolution of the plasma. Calculating the time lag in each pixel of an AIA image can reveal large scale cooling patterns in coronal loops as well as the diffuse emission between loops across an entire \AR{}.

\citet{viall_evidence_2012} computed time lags for all possible AIA EUV channel pairs in every pixel of \AR{} NOAA 11082 and found positive time lags across the entire \AR{} core, indicative of cooling plasma. They interpreted these observations as being inconsistent with a steady heating model. \citet{viall_survey_2017} extended this analysis to the 15 active regions catalogued by \citet{warren_systematic_2012} and found overwhelmingly positive time lags, or cooling plasma, in all cases, with only a few isolated instances of negative time lags, or heating plasma. These observations are consistent with an impulsive heating scenario in which little emission is produced during the heating phase because of the time needed to fill the corona by chromospheric evaporation and the efficiency of thermal conduction. \citet{bradshaw_patterns_2016} predicted AIA intensities for a range of nanoflare heating frequencies in a model \AR{} and applied the time-lag analysis to their simulated images. They found that aspects of both high and intermediate frequency nanoflares reproduced the observed time-lag patterns, but neither model could fully account for the observational constraints, suggestive of a range of heating frequencies across the \AR{}. Additionally, \citet{lionello_can_2016} used a field-aligned hydrodynamic model to compute time lags for several loops in NOAA 11082 and concluded that an impulsive heating model could not account for the long ($>5000$ s) time lags calculated from observations by \citet{viall_evidence_2012}.

Any successful heating model must be able to reproduce the observed distribution of emission measure slopes and time lags. In order to carry out such a test, both advanced forward modeling and sophisticated comparisons to data are required. In this paper, we carry out a series of nanoflare heating simulations in order to better understand how the frequency of impulsive heating events on a given strand is related to observable properties of the plasma, notably the emission measure slope and the time lag as derived from AIA observations. To do this, we use a combination of magnetic field extrapolations, hydrodynamic models, and atomic data to produce simulated AIA emission which can be treated in the same manner as real observations. We then apply the emission meaure and time lag analyses to this simulated data. \autoref{modeling} provides a detailed description of both our forward modeling pipeline and the nanoflare heating model. In \autoref{results}, we show the predicted intensities for each heating model and AIA channel (\autoref{intensities}), the resulting emission measure slopes (\autoref{em_slopes}) and the time lags (\autoref{timelags}). \autoref{discussion} provides some discussion of our results and \autoref{conclusions} includes a summary and concluding remarks.

This paper is the first in a series concerned with constraining nanoflare heating properties through forward modeled observables and serves to describe our forward modeling pipeline and lay out the results of our nanoflare simulations. In \citet[\citetalias{barnes_understanding_2019-1} hereafter]{barnes_understanding_2019-1}, we use machine learning to make detailed comparisons to AIA observations of \AR{} NOAA 1158. We train a random forest classifier using the predicted emission measure slopes and time lags presented here over the entire heating frequency parameter space in order to classify the heating frequency in each pixel of the observed \AR{}. In contrast to past studies which have relied on a single diagnostic, this approach allows us to simultaneously account for an arbitrarily large number of observables in deciding which model fits the data ``best.'' The ability to quantitatively compare models with large quantities of data is crucial for progress in the current era where the amount of solar coronal data is orders of magnitude larger than in the past.  Combined, these two papers demonstrate a novel method for using real and simulated observations to systematically predict heating properties in \AR{} cores.
