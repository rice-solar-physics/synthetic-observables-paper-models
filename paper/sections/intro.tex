%%%%%%%%%%%%%%%%%%%%%%%%%%%%%%%%%%%%%%%%%%%%%%%%%%%%%%%%%%%%%%%%%%%%%%%%%%%%%%%
%                                   Introduction                              %
%%%%%%%%%%%%%%%%%%%%%%%%%%%%%%%%%%%%%%%%%%%%%%%%%%%%%%%%%%%%%%%%%%%%%%%%%%%%%%%
\section{Introduction}\label{introduction}

One property of particular interest is the nanoflare \textit{frequency}, or the rate at which individual events occur on a given coronal strand. In the case of \textit{low-frequency} nanoflares, the time between consecutive events on a given strand is long relative to a characteristic loop cooling time, giving the loop time to fully cool and drain before the loop is reenergized. In the \textit{high-frequency} scenario, the time between events is short relative to the cooling time such that the loop is not allowed to fully cool before being reheated. Steady heating may be regarded as nanoflare heating in the very high-frequency limit. Additional details regarding the parameterization of the heating frequency are given in \autoref{heating}. One possible signature of low-frequency nanoflares is the presence of 8-10 MK plasma in non-flaring active regions, the so-called ``smoking gun'' of nanoflare heating \citep{cargill_implications_1994}. Unfortunately, due to its faint, transient nature, a positive detection of this hot plasma is made difficult by several factors \citep{winebarger_defining_2012,barnes_inference_2016}. However, recent observations of this very hot plasma have provided compelling evidence for the existence of nanoflares \citep[e.g.][]{reale_evidence_2009,schmelz_hinode_2009,testa_hinode/eis_2012,brosius_pervasive_2014,caspi_new_2015,parenti_spectroscopy_2017,ishikawa_detection_2017}.

%Though originally pertaining to energetic events on the order $10^{24}$ erg due to small-scale reconnection, the term \textit{nanoflare} is now synonymous with any impulsive energy release and is not specific to any particular mechanism \citep{klimchuk_key_2015}.

In addition to the presence of ``hot'' plasma, two observables in particular have been used to diagnose heating frequency in observations of \AR{} cores: the emission measure slope and the timelag. \authorcomment1{Need some sort of comment here to differentiate between hot plasma, a direct signature, versus timelag and EM slope which are indirect signatures via cooling; make this more explicit} The emission measure distribution, $\mathrm{EM}(T)=\int\mathrm{d}h\,n_e^2$, where $n_e$ is the electron density and the integration is taken along the line of sight, provides the temperature distribution of the plasma in a single pixel and is a useful diagnostic for parameterizing the underlying heating mechanism. Many observational and theoretical studies have suggested that the ``cool'' portion of the \dem{} (i.e. leftward of the peak, $10^{5.5}\lesssim T\lesssim10^{6.5}$ K), can be described by $\mathrm{EM}(T)\sim T^a$ \citep{jordan_structure_1976,cargill_implications_1994,cargill_nanoflare_2004,warren_systematic_2012}. The so-called \textit{emission measure slope}, $a$, is an important diagnostic for assessing how often a single loop may be reheated and has been used by several workers to interpret \AR{} core observations in terms of both high and low frequency heating \citep[see Table 3 of][and references therin]{bradshaw_diagnosing_2012}. A similar scaling $\mathrm{EM}(T)\sim T^{-b}$ has been claimed for $T>T(\mathrm{EM}_{max})$ though this parameter is difficult to constrain due to the lack of spectral observations in this temperature range \citep{winebarger_defining_2012} as well as the sensitivity of $b$ to temperature bounds of the fit \citep[see section 3.3.1 of][]{barnes_inference_2016-1}.

The ``cool'' \dem{} slope typically falls in the range $2<a<5$, with smaller (larger) values being signatures of low (high) frequency heating. Below $T(\mathrm{EM}_{max})$, \citet{cargill_implications_1994} noted that the \dem{} could be described by $\mathrm{EM}(T)\sim n^2\tau_{rad}$, where $\tau_{rad}\sim T^{1-\alpha}n^{-1}$ is the radiative cooling time. Furthermore, during radiative cooling, \citet{bradshaw_cooling_2010} found that $T\sim n^{\ell}$, with $\ell\approx1$ for long loops and $\ell\approx2$ for short loops. Combining these expressions, $\mathrm{EM}\sim T^{1+1/\ell-\alpha}$, and assuming $\alpha=-1/2$ \citep[i.e. using the radiative losses of][]{rosner_dynamics_1978}, we find $a\approx2$ for short loops and $a\approx2.5$ for long loops in the case of single nanoflares, i.e. low frequency heating. Several modeling studies have examined how $a$ varies with heating frequency. \citet{mulu-moore_can_2011} used a field-aligned hydrodynamic model to simulate single nanoflares localized at the loop apex and found $a\sim2-2.3$ for several different loop lengths and equilibrium temperatures, with slightly lower $a$ for photospheric abundances. \citet{bradshaw_diagnosing_2012} computed synthetic spectral lines from simulations of single nanoflares for a wide range of loop lengths, event durations, and heating rates and found $1.7\le a\le2.6$. In a follow-up study, \citet{reep_diagnosing_2013} using the same method as \citeauthor{bradshaw_diagnosing_2012}, found that multiple consecutive impulsive heating events (i.e. so-called ``nanoflare trains'') could produce a wider range of $a$, $0.88\le a\le4.56$, more consistent with the observed range of slopes. Most recently, \citet{cargill_active_2014}, using the EBTEL model \citep{klimchuk_highly_2008,cargill_enthalpy-based_2012,cargill_enthalpy-based_2012-1}, found that varying the time between consecutive heating events from 250 s (high-frequency heating) to 5000 s (low-frequency heating) could account for the wide observed distribution of $a$, with higher values of $a$ corresponding to higher heating frequency due to the \dem{} distribution becoming increasingly isothermal \citep[see also][]{barnes_inference_2016-1}. 

Many observational studies of active regions have used the \dem{} slope to make conclusions about the heating frequency. \citet{tripathi_emission_2011} computed \dem{} from observations of inter-moss regions using data from the EUV Imaging Spectrometer \citep[EIS,][]{culhane_euv_2007} onboard the \textit{Hinode} spacecraft \citep{kosugi_hinode_2007} and found $a\approx2.4$, consistent with low frequency nanoflares. \citet{warren_constraints_2011} constructed \dem{} distributions using observations of \AR{} cores from the Atmospheric Imaging Assembly \citep[AIA][]{lemen_atmospheric_2012} as well as the X-Ray Telescope \citep[XRT][]{golub_x-ray_2007} and EIS and made comparisons to \dem{} computed from field-aligned hydrodynamic models. \citeauthor{warren_constraints_2011} found $a\approx3.26$ and that their observed \dem{} were most consistent with high-frequency nanoflares though their high-frequency model could not reproduce the emission below $10^{6.5}$ K. Similarly, \citet{winebarger_using_2011} found good agreement between observed \dem{}, computed from XRT and EIS data, and a model \dem{}, computed from a steady heating model, in the temperature range $10^{6.3}<T<10^{6.7}$ K. However, the model \dem{} underestimated the emission in the temperature range $10^{6.1}<T<10^{6.3}$ K, suggesting there may be a low-frequency component to the heating as well. \citet{schmelz_cold_2012} computed \dem{} distributions for 8 different active regions observed by EIS and XRT and found 25\% had $a\sim4-5$, consistent with steady or high-frequency heating, while $\approx60\%$ had $1.91\le a\le2.84$, consistent with low-frequency nanoflares. While all of the studies listed so far have computed the \dem{} from a single pixel or a small area of the \AR{}, \citet{del_zanna_evolution_2015} calculated $a$ in every pixel of the \AR{} core from from AIA and EIS observations of NOAA 11193. \citeauthor{del_zanna_evolution_2015} found higher values of $a$ near the center of the core and lower values near the outer edge of the core, suggesting a distribution of heating frequencies across the \AR{}. \authorcomment1{Point out problems of only using EM slopes to motivate use of timelag}

%While the emission measure slope is a convenient parameterization of the heating frequency, there are drawbacks to this method, including the lack of unique solution to the \dem{} inversion problem \citet[see][]{aschwanden_benchmark_2015} as well as multiple emitting structures along the line-of-sight (LOS) and uncertainties in the atomic data. Furthermore, $a$ in a single pixel or small area of an \AR{} may not necessarily be representative of the whole \AR{} \citep[e.g.][]{del_zanna_evolution_2015}.

In addition to the \dem{} slope, the timelag analysis of \citet{viall_evidence_2012} has recently been used by many workers in both observational and modeling studies to understand the frequency of energy release in \AR{} cores. The \textit{timelag} is the temporal delay which maximizes the cross-correlation between two timeseries, and, qualitatively, can be thought of as the amount of time which one signal must be shifted relative to another in order to achieve the best ``match'' between the two signals. When applied to data from the 6 EUV channels of AIA, the timelag can be used as a proxy for the cooling time between channels and can reveal large-scale cooling patterns in coronal loops across an entire \AR{}. \citeauthor{viall_evidence_2012} computed timelags for all possible AIA EUV channel pairs in every pixel of \AR{} NOAA 11082 and found positive timelags across the entire \AR{} core, indicative of cooling plasma. They interpreted these observations as being inconsistent with a steady heating model. \citet{viall_survey_2017} extended this analysis to the 15 active regions catalogued by \citet{warren_systematic_2012} and found overwhelmingly positive timelags in all cases, indicative of cooling plasma and consistent with nanoflare heating. \citet{bradshaw_patterns_2016} forward modeled AIA intensities for a range of nanoflare heating frequencies and applied the timelag analysis to these synthetic images. They found that both high and intermediate frequency nanoflares reproduced the observed timelag patterns and were suggestive of a range of heating frequencies across the \AR{}.  Additionally, \citet{lionello_can_2016}, \citet{winebarger_investigation_2016}, and most recently \citet{winebarger_identifying_2018} have found that thermal non-equilibrium cycles, driven by highly-stratified steady footpoint heating, can produce timelag signatures similar to those of impulsive heating models, suggesting that observed timelags may be consistent with both impulsive and steady heating. \authorcomment1{Summarize current findings; motivate this study}

In this paper, we carry out a series of nanoflare heating simulations in order to better understand how the frequency of impulsive heating events on a given strand is related to observable properties of the plasma, notably the emission measure slope and the timelag as derived from AIA observations. To carry out this study, we use a combination of magnetic field extrapolations, hydrodynamic models, atomic data, and advanced forward modeling to simulate EUV observations from SDO/AIA. The resulting forward modeled intensities can then be treated in the same manner as actual observations and we carry out the emission meaure distribution and timelag analysis as such. \autoref{modeling} provides a detailed description of our forward modeling pipeline, including the parameterization of the heating model. In \autoref{results}, we show the results of the emission measure (\autoref{em_slopes}) and timelag (\autoref{timelags}) analysis as applied to our simulated intensity maps. \autoref{discussion} provides some discussion of our results and \autoref{conclusions} provides a summary and concluding remarks. Additionally, \autoref{nei} shows how to compute ion populations while accounting for non-equilibrium ionization and \autoref{timelag_details} details our efficient and scalable method for computing the timelag between intensity timeseries in two different channels of AIA.

This paper is the first in a series of papers concerned with constraining nanoflare heating properties through forward modeled observables and serves primarily to describe our forward modeling pipeline and lay out the results of our nanoflare simulations. In \citet[\citetalias{barnes_understanding_2018-1} hereafter]{barnes_understanding_2018-1}, we make detailed comparisons to observations and use the model results presented here to classify the heating frequency in each pixel of the observed \AR{}. Combined, these two papers demonstrate a novel method for using data, machine learning, and forward modeling to systematically predict heating properties in \AR{} cores.
