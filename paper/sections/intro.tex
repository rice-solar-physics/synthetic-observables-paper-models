%%%%%%%%%%%%%%%%%%%%%%%%%%%%%%%%%%%%%%%%%%%%%%%%%%%%%%%%%%%%%%%%%%%%%%%%%%%%%%%
%                                   Introduction                              %
%%%%%%%%%%%%%%%%%%%%%%%%%%%%%%%%%%%%%%%%%%%%%%%%%%%%%%%%%%%%%%%%%%%%%%%%%%%%%%%
\section{Introduction}\label{introduction}

Nanoflares, small-scale impulsive events, have long been used to explain the observed million-degree temperatures in the non-flaring corona. Though originally pertaining to energetic bursts of order $10^{24}$ erg resulting from small-scale reconnection \citep{parker_nanoflares_1988}, the term \textit{nanoflare} is now synonymous with any impulsive energy release and is not specific to any particular mechanism \citep{klimchuk_key_2015}. Due to their faint, transient nature, direct observations of nanoflares are made difficult by several factors, including inadequate spectral coverage of instruments, the efficiency of thermal conduction, and non-equilibrium ionization \citep{cargill_implications_1994,winebarger_defining_2012,barnes_inference_2016}. However, recent observations of ``very hot'' 8-10 MK plasma, the so-called ``smoking gun'' of nanoflares, have provided compelling evidence for their existence \citep[e.g.][]{brosius_pervasive_2014,caspi_new_2015,parenti_spectroscopy_2017,ishikawa_detection_2017}

Critical to understanding the underlying heating mechanism is knowing whether coronal loops in non-flaring active regions are heated \textit{steadily} or \textit{impulsively}, or, more precisely, at what frequency do nanflares repeat on a given coronal loop. In the case of low-frequency nanoflares, the time between consecutive events on a loop is long relative to its characteristic cooling time, giving the loop time to fully cool and drain before it is reenergized. In the high-frequency scenario, the time between events is short relative to the cooling time such that the loop is not allowed to fully cool before being heated again. Steady heating may be regarded as nanoflare heating in the very high-frequency limit.

In lieu of a direct observable signature of nanoflare heating, two parameters in particular have been used to diagnose heating frequency in \AR{} cores: the emission measure slope and the timelag. These diagnostics provide \textit{indirect} signatures of the energy deposition via observations of the plasma cooling by thermal conduction and radiation. We will now discuss each of these observables in detail.

The emission measure distribution, $\mathrm{EM}(T)=\int\mathrm{d}h\,n_e^2$, where $n_e$ is the electron density and the integration is taken along the line of sight, is a useful diagnostic for parameterizing the frequency of energy deposition. Many observational and theoretical studies have suggested that the ``cool'' portion of the \dem{} (i.e. leftward of the peak, $10^{5.5}\lesssim T\lesssim10^{6.5}$ K), can be described by $\mathrm{EM}(T)\sim T^a$ \citep{jordan_structure_1976,cargill_implications_1994,cargill_nanoflare_2004}. The so-called \textit{emission measure slope}, $a$, is an important diagnostic for assessing how often a single loop may be reheated and has been used by several workers to interpret \AR{} core observations in terms of both high and low frequency heating \citep[see Table 3 of][and references therin]{bradshaw_diagnosing_2012}. The ``cool'' emission measure slope typically falls in the range $2<a<5$, with smaller (larger) values being signatures of low (high) frequency heating. Many observational studies of active region cores have used the emission measure slope to make conclusions about the heating frequency \citep{tripathi_emission_2011,warren_constraints_2011,winebarger_using_2011,schmelz_cold_2012,warren_systematic_2012,del_zanna_evolution_2015}.

To better understand observable properties of nanoflare heating, several workers have used hydrodynamic models of coronal loops to examined how the emission measure slope varies with heating frequency \citep{mulu-moore_can_2011,bradshaw_diagnosing_2012,reep_diagnosing_2013}. Most recently, \citet{cargill_active_2014} found that varying the time between consecutive heating events from 250 s (high-frequency heating) to 5000 s (low-frequency heating) could account for the wide observed distribution of emission measure slopes, with higher values of $a$ corresponding to higher heating frequency due to the \dem{} distribution becoming increasingly isothermal \citep[see also][]{barnes_inference_2016-1}.

In addition to the emission measure slope, the timelag analysis of \citet{viall_evidence_2012} has been used by several workers to understand the frequency of energy release in \AR{} cores. The \textit{timelag} is the temporal delay which maximizes the cross-correlation between two timeseries, and, qualitatively, can be thought of as the amount of time which one signal must be shifted relative to another in order to achieve the best ``match'' between the two signals. As the plasma cools through the six EUV channels of the Atmospheric Imaging Assembly \citep[AIA,][]{lemen_atmospheric_2012} onboard the Solar Dynamics Observatory spacecraft \citep[SDO,][]{pesnell_solar_2012}, we expect to see the intensity peak in successively cooler passbands of AIA according to the sensitivity of each channel in temperature space. Computing the timelag between lightcurves in different channels provides a proxy for the cooling time between channels and insight into the thermal evolution of the plasma. Computing the timelag in each pixel of an AIA image can reveal large scale cooling patterns in coronal loops across an entire \AR{}.

\citeauthor{viall_evidence_2012} computed timelags for all possible AIA EUV channel pairs in every pixel of \AR{} NOAA 11082 and found positive timelags across the entire \AR{} core, indicative of cooling plasma. They interpreted these observations as being inconsistent with a steady heating model. \citet{viall_survey_2017} extended this analysis to the 15 active regions catalogued by \citet{warren_systematic_2012} and found overwhelmingly positive timelags in all cases. \citet{bradshaw_patterns_2016} forward modeled AIA intensities for a range of nanoflare heating frequencies and applied the timelag analysis to these synthetic images. They found that both high and intermediate frequency nanoflares reproduced the observed timelag patterns and were suggestive of a range of heating frequencies across the \AR{}.

Any successful heating model must be able to reproduce the observed distribution of emission measure slopes and timelags. In order to carry out such a test, both advanced forward modeling and sophisticated comparisons to data are required. In this paper, we carry out a series of nanoflare heating simulations in order to better understand how the frequency of impulsive heating events on a given strand is related to observable properties of the plasma, notably the emission measure slope and the timelag as derived from AIA observations. To do this, we use a combination of magnetic field extrapolations, hydrodynamic models, and atomic data to create synthetic AIA images which can then be treated in the same manner as real observations. We then apply the emission meaure and timelag analysis to this simulated data. \autoref{modeling} provides a detailed description of each step of our forward modeling pipeline and the loop heating model. In \autoref{results}, we show the synthetic intensities for each heating model and AIA channel (\autoref{intensities}), the resulting emission measure slopes (\autoref{em_slopes}), and timelags (\autoref{timelags}) and discuss their significance. \autoref{conclusions} provides a summary and concluding remarks.

This paper is the first in a series of papers concerned with constraining nanoflare heating properties through forward modeled observables and serves to describe our forward modeling pipeline and lay out the results of our nanoflare simulations. In \citet[\citetalias{barnes_understanding_2018-1} hereafter]{barnes_understanding_2018-1}, we make detailed comparisons to observations and use the model results presented here to classify the heating frequency in each pixel of the observed \AR{}. Combined, these two papers demonstrate a novel method for using real and synthetic observations to systematically predict heating properties in \AR{} cores.
