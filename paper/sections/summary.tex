%%%%%%%%%%%%%%%%%%%%%%%%%%%%%%%%%%%%%%%%%%%%%%%%%%%%%%%%%%%%%%%%%%%%%%%%%%%%%%%
%                                   Summary and Conclusions                   %
%%%%%%%%%%%%%%%%%%%%%%%%%%%%%%%%%%%%%%%%%%%%%%%%%%%%%%%%%%%%%%%%%%%%%%%%%%%%%%%

\section{Summary}\label{conclusions}

We have carried out a series of numerical simulations in an effort to understand how signatures of the nanoflare heating frequency are manifested in two observables: the emission measure slope and the time lag. Additionally, we described each component of our pipeline for forward modeling \AR{} emission. For a given magnetogram observation of the relevant \AR{} (in this case, NOAA 1158), we compute a potential field extrapolation and trace a large number of field lines through the extrapolated vector field. For each traced field line, we run an EBTEL hydrodynamic simulation and use the resulting temperatures and densities, combined with data from CHIANTI and the instrument response function, to compute the time-dependent intensity. These intensities are then mapped back to the magnetic skeleton and integrated along the LOS in each pixel to create time-dependent images of the \AR{}.

Using our novel and efficient forward modeling pipeline, we produced AIA images for all six EUV channels for $\approx8$ hours of simulation time. From these results, we computed both the emission measure slope and the time lag for all possible channel pairs. We carried out these steps for three different nanoflare heating frequencies, high, intermediate, and low, (see \autoref{eq:heating_types}) in addition to two control models, for a total of five different heating scenarios (see \autoref{tab:heating}).

Our results can be summarized in the following points:
\begin{enumerate}
    \item As the heating frequency decreases, the emission measure slope, $a$, becomes increasingly shallow, saturating at $a\approx2$. As the heating frequency increases, the distribution of slopes over the \AR{} is shifted to higher values and broadens.
    \item The time lag becomes increasingly spatially coherent with decreasing heating frequency. When strands are allowed to cool without being re-energized, the spatial distribution of time lags is largely determined by the distribution of loop lengths over the \AR{}.
    \item The distribution of time lags becomes increasingly broad as the heating frequency increases, consistent with the results of \citet{viall_signatures_2016}.
    \item Negative time lags in channel pairs where the second (``cool'') channel is 131 \AA{} provide a possible diagnostic for $\ge10$ MK plasma
\end{enumerate}

In this paper, we have used our advanced forward modeling pipeline to systematically examine how the emission measure slope and time lag are affected by the nanoflare heating frequency. In \citetalias{barnes_understanding_2019-1}, we use the model results presented here to train a random forest classifier and apply it to emission measure slopes and time lags derived from real AIA observations of NOAA 1158. The 15 channel pairs for the time lag and cross-correlation combined with the emission measure slope represent a 31-dimensional feature space and a single 500-by-500 pixel \AR{} amounts to $2.5\times10^5$ sample points. Performing an accurate assessment over this amount of data manually or ``by eye'' is at least impractical and likely impossible. Thus, the application of machine learning to the problem of assessing models in the context of real data is a critical step in understanding the underlying energy deposition in \AR{} cores and, to our knowledge, has not yet been applied in this context.  
