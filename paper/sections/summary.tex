%%%%%%%%%%%%%%%%%%%%%%%%%%%%%%%%%%%%%%%%%%%%%%%%%%%%%%%%%%%%%%%%%%%%%%%%%%%%%%%
%                                   Summary and Conclusions                   %
%%%%%%%%%%%%%%%%%%%%%%%%%%%%%%%%%%%%%%%%%%%%%%%%%%%%%%%%%%%%%%%%%%%%%%%%%%%%%%%
\section{Summary and Conclusions}\label{conclusions}

We have carried out a series of numerical simulations in an effort to understand how signatures of the nanoflare heating frequency are manifested in two observables: the emission measure slope and the timelag. Additionally, we described each component of our pipeline for forward modeling \AR{} emission. For a given magnetogram observation of the relevant \AR{} (in this case, NOAA 1158), we compute a potential field extrapolation and trace a large number of fieldlines through the extrapolated vector field. For each traced fieldline, we run an EBTEL hydrodynamic simulation and use the resulting temperatures and densities, combined with data from CHIANTI and the instrument response function, to compute synthetic time-dependent intensities. Using our magnetic skeleton, these intensities are then integrated along the LOS in each pixel to create time-dependent images of the \AR{}.

Using our novel and efficient forward modeling pipeline, we produced synthetic AIA images for all 6 EUV channels for a total of $\approx8$ hours of simulation time over an entire \AR{}. From these synthetic images, we computed both the emission measure slope and the timelag for all possible channel pairs. We carried out these steps for three different nanoflare heating frequencies, high, intermediate, and low, (see \autoref{eq:heating_types}) in addition to two control models, for a total of five different heating scenarios (see \autoref{tab:heating}).

Our results can be summarized in the following points:
\begin{enumerate}
    \item As the heating frequency decreases, the emission measure slope, $a$, becomes increasingly shallow, saturating at $a\approx2$. As the heating frequency increases, the distribution of slopes over the \AR{} is shifted to higher values and broadens.
    \item The timelag becomes increasingly spatially coherent with decreasing heating frequency. When loops are allowed to cool without being reenergized, the spatial distribution of timelags is largely determined by the distribution of loop lengths over the \AR{}.
    \item The distribution of timelags becomes increasingly broad and approaches a uniform distribution as the heating frequency increases, consistent with the results of \citet{viall_signatures_2016}.
\end{enumerate}

Though we have not addressed it here, another possible mechanism for producing time-varying intensity in \AR{} loops is thermal non-equilibrium (TNE) wherein condensation cycles driven by highly-stratified, but steady footpoint heating lead to long-period intensity pulsations \citep{kuin_thermal_1982}. Though originally used to explain coronal rain \citep{antolin_coronal_2010,antolin_multithermal_2015,auchere_coronal_2018} and prominences \citep{antiochos_model_1991}, several workers \citep{lionello_can_2016,winebarger_investigation_2016,froment_long-period_2017,winebarger_identifying_2018,froment_occurrence_2018} have recently claimed that TNE can produce timelag signatures similar to those of impulsive heating models, suggesting that observed timelags may be consistent with both impulsive and steady heating. However, it is not yet clear whether TNE is consistent with observed signatures of very hot (8-10 MK) plasma. Detailed comparisons between TNE and nanoflare simulations and observations are desparately needed.

In this paper, we have used our advanced forward modeling pipeline to systematically examine how the emission measure slope and timelag are affected by the nanoflare heating frequency. In \citetalias{barnes_understanding_2018-1}, we use the model results presented here to train a random forest classification model and apply it to emission measure slopes and timelags derived from real AIA observations of NOAA 1158. This represents a novel method for making comparisons between simulations and real data and is a critical step in understanding the distribution of heating frequencies in \AR{} cores.
