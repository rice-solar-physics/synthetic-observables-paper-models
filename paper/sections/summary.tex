%%%%%%%%%%%%%%%%%%%%%%%%%%%%%%%%%%%%%%%%%%%%%%%%%%%%%%%%%%%%%%%%%%%%%%%%%%%%%%%
%                                   Summary and Conclusions                   %
%%%%%%%%%%%%%%%%%%%%%%%%%%%%%%%%%%%%%%%%%%%%%%%%%%%%%%%%%%%%%%%%%%%%%%%%%%%%%%%
\section{Summary and Conclusions}\label{conclusions}

\authorcomment1{summarize tool presented and what we did with it}

\authorcomment1{Offer concluding points: both slopes and timelag preserve signatures of heating, distrubution of EM slopes shifts to lower a and narrows with decreasing frequency, timelag maps increasingly spatially coherent with decreasing frequency, timelags show clear dependence on both loop length}

\authorcomment1{future work: hot plasma, spectral diagnostics}

\authorcomment1{TNE another possibility for driving variability in EUV intensity and producing timelags and EM slopes, but hot plasma?}

\authorcomment1{Tease comparisons with observations included in paper 2}
