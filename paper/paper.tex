%% Use AASTeX class, version 6.1
%% Allow for additional class options such as,
%%  twocolumn   : two text columns, 10 point font, single spaced article.
%%                This is the most compact and represent the final published
%%                derived PDF copy of the accepted manuscript from the publisher
%%  manuscript  : one text column, 12 point font, double spaced article.
%%  preprint    : one text column, 12 point font, single spaced article.  
%%  preprint2   : two text columns, 12 point font, single spaced article.
%%  modern      : a stylish, single text column, 12 point font, article with
%% 		            wider left and right margins. This uses the Daniel
%% 		            Foreman-Mackey and David Hogg design.
%%  astrosymb    : Loads Astrosymb font and define \astrocommands. 
%%  tighten      : Makes baselineskip slightly smaller, only works with 
%%                 the twocolumn substyle.
%%  times        : uses times font instead of the default
%%  linenumbers  : turn on lineno package.
%%  trackchanges : required to see the revision mark up and print its output
%%  longauthor   : Do not use the more compressed footnote style (default) for 
%%                 the author/collaboration/affiliations. Instead print all
%%                 affiliation information after each name. Creates a much
%%                 long author list but may be desirable for short author papers

\documentclass[twocolumn,linenumbers]{aastex62}
%% Include packages
\usepackage{amsmath}
\usepackage{calc}
\usepackage{pgf}
\usepackage{pythontex}
% % % % % % % % % % % % % % % % % % % % % % % % % % % % % % % % % %
% PythonTeX Bug Fix % % % % % % % % % % % % % % % % % % % % % % % %
% % % % % % % % % % % % % % % % % % % % % % % % % % % % % % % % % % 
% pytexbug fix for context in customcode.
\makeatletter
\renewenvironment{pythontexcustomcode}[2][begin]{%
	\VerbatimEnvironment
	\Depythontex{env:pythontexcustomcode:om:n}%
	\ifstrequal{#1}{begin}{}{%
		\ifstrequal{#1}{end}{}{\PackageError{\pytx@packagename}%
			{Invalid optional argument for pythontexcustomcode}{}
		}%
	}%
	\xdef\pytx@type{CC:#2:#1}%
	\edef\pytx@cmd{code}%
	% PATCH \def\pytx@context{}%
	\pytx@SetContext
	% END PATCH
	\def\pytx@group{none}%
	\pytx@BeginCodeEnv[none]}%
{\end{VerbatimOut}%
\setcounter{FancyVerbLine}{\value{pytx@FancyVerbLineTemp}}%
\stepcounter{\pytx@counter}%
}%
\makeatother
% % % % % % % % % % % % % % % % % % % % % % % % % % % % % % % % % %
\setpythontexcontext{textwidth=\the\textwidth,columnwidth=\the\columnwidth,figurewidth=\the\columnwidth}
%% Custom commands
\DeclareMathOperator*{\argmax}{arg\,max} % in your preamble
\newcommand{\AR}{active region}
\newcommand{\dem}{$\mathrm{EM}(T)$}
\renewcommand{\sectionautorefname}{Section}
\renewcommand{\subsectionautorefname}{Section}
\renewcommand{\subsubsectionautorefname}{Section}
%% Paper Aliases
\defcitealias{barnes_understanding_2018-1}{Paper II}
%%%%%%%%%%%%%%%%%%%%%%%%%%%%%%%%%%%%%%%%%%%%%%%%%%%%%%%%%%%%%%%%%%%%%%%%%%%%%%%
%                                   Body                                      %
%%%%%%%%%%%%%%%%%%%%%%%%%%%%%%%%%%%%%%%%%%%%%%%%%%%%%%%%%%%%%%%%%%%%%%%%%%%%%%%
\begin{document}
% TeXFigure Manager
\begin{pythontexcustomcode}{py}
import os
import texfigure
pytex.formatter = texfigure.repr_latex_formatter
import numpy as np
import h5py
import pandas as pd
import matplotlib.pyplot as plt
import matplotlib.colors
from sunpy.map import Map
import astropy.units as u
from astropy.coordinates import SkyCoord
from astropy.visualization import ImageNormalize,AsinhStretch,SqrtStretch
import plasmapy.atomic
import synthesizAR
from synthesizAR.atomic import EmissionModel
# Set some plot styling here
plt.rcParams['font.family'] = 'serif'
plt.rcParams['font.serif'] = 'Computer Modern'
plt.rcParams['text.usetex'] = True
plt.rcParams['axes.titlesize'] = 10
plt.rcParams['axes.labelsize'] = 10
plt.rcParams['legend.fontsize'] = 8
plt.rcParams['xtick.labelsize'] = 8
plt.rcParams['ytick.labelsize'] = 8
plt.rcParams['xtick.major.pad'] = 8
plt.rcParams['xtick.minor.pad'] = 8
plt.rcParams['ytick.major.pad'] = 8
plt.rcParams['ytick.minor.pad'] = 8
plt.rcParams['xtick.direction'] = 'in'
plt.rcParams['ytick.direction'] = 'in'
plt.rcParams['savefig.dpi'] = 200
plt.rcParams['savefig.format'] = 'pdf'
plt.rcParams['savefig.bbox'] = 'tight'
# Some useful quantities
channels = [94,131,171,193,211,335]
heating = ['high_frequency', 'intermediate_frequency','low_frequency']
channel_pairs = [(94,335), (94,171), (94,193),(94,131),(94,211),(335,131),(335,193),
                 (335,211),(335,171),(211,131),(211,171),(211,193),(193,171),(193,131),
                 (171,131),]
selected_channel_pairs = [channel_pairs[i] for i in (0, 9, 12)]
\end{pythontexcustomcode}
%%%%%%%%%%%%%%%%%%%%%%%%%%%%%%%%%%%%%%%%%%%%%%%%%%%%%%%%%%%%%%%%%%%%%%%%%%%%%%%
%                                   Title and Authors                         %
%%%%%%%%%%%%%%%%%%%%%%%%%%%%%%%%%%%%%%%%%%%%%%%%%%%%%%%%%%%%%%%%%%%%%%%%%%%%%%%
\title{Understanding Heating Frequency in Active Region Cores through Synthetic Observables I. Modeling}
\author[0000-0001-9642-6089]{W. T. Barnes}
\author{S. J. Bradshaw}
\affiliation{Department of Physics \& Astronomy, Rice University, Houston, TX 77005-1827}
\author{N. M. Viall}
\affiliation{NASA Goddard Space Flight Center, Greenbelt, MD 20771}
\correspondingauthor{W. T. Barnes}
\email{will.t.barnes@rice.edu}
%%%%%%%%%%%%%%%%%%%%%%%%%%%%%%%%%%%%%%%%%%%%%%%%%%%%%%%%%%%%%%%%%%%%%%%%%%%%%%%
%                                   Abstract                                  %
%%%%%%%%%%%%%%%%%%%%%%%%%%%%%%%%%%%%%%%%%%%%%%%%%%%%%%%%%%%%%%%%%%%%%%%%%%%%%%%
\begin{abstract}
A key question surrounding the coronal heating problem is the frequency at which the coronal plasma is reheated. In the context of active regions, there is a longstanding debate regarding whether individual loops are heated steadily or impulsively. To properly address this question detailed models and systematic comparisons to the data are required. In this paper, we describe a pipeline for forward modeling emission from active regions using field extrapolations and hydrodynamics loop models. We use this pipeline to model time-dependent active region intensities as observed by SDO/AIA for a range of nanoflare heating frequencies. From our synthetic EUV data, we compute the emission measure slope and timelag and find that signatures of the heating frequency persist in both of these parameters. In a follow up paper, we will use these synthetic observables to train a machine learning classification model in order to classify real AIA data in terms of the heating frequency.
\end{abstract}
%% Keywords
\keywords{Sun,corona,nanoflares,active regions}
%%%%%%%%%%%%%%%%%%%%%%%%%%%%%%%%%%%%%%%%%%%%%%%%%%%%%%%%%%%%%%%%%%%%%%%%%%%%%%%
%                                   Sections                                  %
%%%%%%%%%%%%%%%%%%%%%%%%%%%%%%%%%%%%%%%%%%%%%%%%%%%%%%%%%%%%%%%%%%%%%%%%%%%%%%%
%%%%%%%%%%%%%%%%%%%%%%%%%%%%%%%%%%%%%%%%%%%%%%%%%%%%%%%%%%%%%%%%%%%%%%%%%%%%%%%
%                                   Introduction                              %
%%%%%%%%%%%%%%%%%%%%%%%%%%%%%%%%%%%%%%%%%%%%%%%%%%%%%%%%%%%%%%%%%%%%%%%%%%%%%%%
\section{Introduction}\label{introduction}

%\authorcomment1{Start with a few sentences about coronal heating; move into short discussion on frequency; then discuss how frequency might be diagnosed with lots of background on previous studies of heating frequency in active regions; Provide some background on the timelag method; how does this help us understand the heating frequency?}

\authorcomment2{Need a few lead-in sentences here.} First proposed by \citet{parker_nanoflares_1988}, nanoflare heating provides a possible mechanism for heating the solar corona through small-scale, bursty energy release. Though originally pertaining to energetic events on the order $10^{24}$ erg due to small-scale reconnection, the term \textit{nanoflare} is now synonymous with any impulsive energy release and is not specific to any particular mechanism \citep{klimchuk_key_2015}. 

%\authorcomment1{Hot plasma possible signature; evidence from EUNIS, FOXSI, Parenti; but hot plasma difficult to detect, see Winebarger, Barnes, others?; cool slopes and timelags as an alternative though more indirect signatures of nanoflares}

While nanoflares remain a viable coronal heating mechanism, identifying observational signatures of impulsive heating events is non-trivial. One such signature is the presence of 8-10 MK plasma in non-flaring active regions, the so-called ``smoking gun'' of nanoflare heating \citep{cargill_implications_1994}. Unfortunately, due to its faint, transient nature, a positive detection of this hot plasma is made difficult by several factors \citep{winebarger_defining_2012,barnes_inference_2016}. However, recent observations of this very hot plasma have provided compelling evidence for the existence of nanoflares \citep[e.g.][]{reale_evidence_2009,schmelz_hinode_2009,testa_hinode/eis_2012,brosius_pervasive_2014,caspi_new_2015,parenti_spectroscopy_2017,ishikawa_detection_2017}. \authorcomment2{Maybe expand a bit on a few of these papers}.

Observations provide critical information for constraining nanoflare heating properties. One property of particular interest is the nanoflare \textit{frequency}, or the rate at which individual events occur on a given coronal strand. In the case of \textit{low-frequency} nanoflares, the time between consecutive events on a given strand is long relative to a characteristic loop cooling time, giving the loop time to fully cool and drain before the next event begins. In the \textit{high-frequency} scenario, the time between events is short relative to the cooling time such that the loop is not allowed to fully cool before being reheated. Steady heating may be regarded as nanoflare heating in the very high-frequency limit. Additional details regarding the parameterization of the heating frequency are given in \autoref{heating}. In addition to the presence of ``hot'' plasma, two observables in particular have been used to diagnose heating frequency in observations of \AR{} cores: the emission measure slope and the timelag. 

The emission measure distribution, $\mathrm{EM}(T)=\int\mathrm{d}h\,n^2$, provides the temperature distribution of the plasma in a single pixel and is a useful diagnostic for parameterizing the underlying heating mechanism. Many observational and theoretical studies have suggested that the ``cool'' portion of the \dem{} (i.e. leftward of the peak, $10^{5.5}\lesssim T\lesssim10^{6.5}$ K), can be described by $\mathrm{EM}(T)\sim T^a$ \citep{jordan_structure_1976,cargill_implications_1994,cargill_nanoflare_2004,warren_systematic_2012}. The so-called \textit{emission measure slope}, $a$, is an important diagnostic for assessing how often a single loop may be reheated and has been used by several workers to interpret \AR{} core observations in terms of both high and low frequency heating \citet[see Table 3 of][and references therin]{bradshaw_diagnosing_2012}. A similar scaling $\mathrm{EM}(T)\sim T^{-b}$ has been claimed for $T>\argmax_T\,\mathrm{EM}$ though this parameter is very sensitive to the temperature range over which the fit is performed \citep[see section 3.3.1 of][]{barnes_inference_2016-1}.

The ``cool'' \dem{} slope typically falls in the range $2<a<5$, with smaller (larger) values being signatures of low (high) frequency heating. Below $T(\mathrm{EM}_{max})$, \citet{cargill_implications_1994} noted that the \dem{} could be described by $\mathrm{EM}(T)\sim n^2\tau_{rad}$, where $\tau_{rad}\sim T^{1-\alpha}n^{-1}$ is the radiative cooling time. Furthermore, during radiative cooling, \citet{bradshaw_cooling_2010} found that $T\sim n^{\ell}$, with $\ell\approx1$ for long loops and $\ell\approx2$ for short loops. Combining these expressions, $\mathrm{EM}\sim T^{1+1/\ell-\alpha}$. Thus, assuming $\alpha=-1/2$ \citep[i.e. using the radiative losses of][]{rosner_dynamics_1978}, we find $a\approx2$ for short loops and $a\approx2.5$ for long loops in the case of single nanoflares, i.e. low frequency heating. Higher values of $a$ are expected as the heating frequency increases due to the \dem{} distribution becoming increasingly isothermal \citep{cargill_active_2014}. \authorcomment2{Add discussion of the rest of the modeling studies here, \citep[][and references therein]{mulu-moore_can_2011,bradshaw_diagnosing_2012,reep_diagnosing_2013,cargill_active_2014}}

Many observational studies of active regions have used the \dem{} slope to make conclusions about the heating frequency. \citet{tripathi_emission_2011} computed \dem{} from observations of inter-moss regions using data from the EUV Imaging Spectrometer \citep[EIS][]{culhane_euv_2007} onboard the \textit{Hinode} spacecraft \citep{kosugi_hinode_2007} and found $a\approx2.4$, consistent with low frequency nanoflares. \citet{warren_constraints_2011} constructed \dem{} distributions using observations of \AR{} cores from the Atmospheric Imaging Assembly \citep[AIA][]{lemen_atmospheric_2012} as well as the X-Ray Telescope \citep[XRT][]{golub_x-ray_2007} and EIS and made comparisons to \dem{} computed from field-aligned hydrodynamic models. \citeauthor{warren_constraints_2011} found $a\approx3.26$ and that their observed \dem{} were most consistent with high-frequency nanoflares though their high-frequency model could not reproduce the emission below $10^{6.5}$ K. Similarly, \citet{winebarger_using_2011} found good agreement between observed \dem{}, computed from XRT and EIS data, and a model \dem{}, computed from a steady heating model, in the temperature range $10^{6.3}<T<10^{6.7}$ K. However, the model \dem{} underestimated the emission in the temperature range $10^{6.1}<T<10^{6.3}$ K, suggesting there may be a low-frequency component to the heating as well. \citet{schmelz_cold_2012} computed \dem{} distributions for 8 different active regions observed by EIS and XRT and found 25\% had $a\sim4-5$, consistent with steady or high-frequency heating, while $\approx60\%$ had $1.91\le a\le2.84$, consistent with low-frequency nanoflares. While all of the studies listed so far have computed the \dem{} from a single pixel or a small area of the \AR{}, \citet{del_zanna_evolution_2015} calculated $a$ in every pixel of the \AR{} core from from AIA and EIS observations of NOAA 11193. \citeauthor{del_zanna_evolution_2015} found higher values of $a$ near the center of the core and lower values near the outer edge of the core, suggesting a distribution of heating frequencies across the \AR{}.

In general, deriving a \dem{} from observations is non-trivial and requires the use of an inversion method with a non-unique solution \citet[see][]{aschwanden_benchmark_2015}. Interpretation of these results is made difficult by several factors including multiple polluting structures along the LOS, uncertainties in the atomic data, and inadequate spectral coverage \citep{winebarger_defining_2012}.

\authorcomment1{Same as above but for timelags; then, thorough discussion of findings of authors using timelag method in the context of high/steady or low frequency heating}

\citet{viall_transition_2015,viall_survey_2017,bradshaw_patterns_2016,lionello_can_2016,winebarger_investigation_2016,winebarger_identifying_2018}

%\authorcomment1{Need to connect previous paragraph to frequency and how observables can be used to diagnose heating frequency; lead into impulsive v steady debate and cite MANY previous studies; these signatures not only provide a confirmation of the nanoflares existence, but contain information regarding nanoflare properties as well; }

In this paper, we carry out a series of nanoflare heating simulations in order to better understand how the frequency of impulsive heating events on a given strand is related to observable properties of the plasma, notably the emission measure slope and the timelag as derived from AIA observations. To carry out this study, we use a combination of magnetic field extrapolations, hydrodynamic models, atomic data, and advanced forward modeling to simulate EUV observations from SDO/AIA. The resulting forward modeled intensities can then be treated in the same manner as actual observations and we carry out the emission meaure and timelag analysis as such. \autoref{modeling} provides a detailed description of our forward modeling pipeline, including the parameterization of the heating model. In \autoref{results}, we show the results of the timelag (\autoref{timelags}) and emission measure (\autoref{em_slopes}) analysis as applied to our simulated intensity maps. \autoref{discussion} provides some discussion of our results and \autoref{conclusions} provides a summary and concluding remarks. Additionally, \autoref{nei} shows how to compute ion populations while accounting for non-equilibrium ionization and \autoref{timelag_details} details our efficient and scalable method for computing the timelag between intensity timeseries in two different channels of AIA.

This paper is the first in a series of papers concerned with constraining nanoflare heating properties through forward modeled observables and serves primarily to describe our forward modeling pipeline and lay out the simulation results. In \citetalias{barnes_understanding_2018-1}, we make detailed comparisons to observations and use the model results presented here to classify the heating frequency in each pixel of the observed \AR{}. Combined, these two papers demonstrate a novel method for using data and forward modeled observables to make sysematic predictions of heating properties in \AR{} cores.

%%%%%%%%%%%%%%%%%%%%%%%%%%%%%%%%%%%%%%%%%%%%%%%%%%%%%%%%%%%%%%%%%%%%%%%%%%%%%%%
%                                   Methods                                   %
%%%%%%%%%%%%%%%%%%%%%%%%%%%%%%%%%%%%%%%%%%%%%%%%%%%%%%%%%%%%%%%%%%%%%%%%%%%%%%%
\section{Modeling}\label{modeling}

In order to understand how signatures of the heating frequency are manifested in the emission measure slope and timelag, we produce synthetic images for a range of nanoflare heating frequencies. To do this, we have constructed an advanced forward modeling pipeline for producing synthetic \AR{} emission from a combination of field extrapolations, hydrodynamic simulations, and atomic data. In the following section, we discuss each step of our pipeline in detail.

\begin{pycode}[manager_methods]
manager_methods = texfigure.Manager(
    pytex, './',
    python_dir='python',
    fig_dir='figures',
    data_dir='data'
)
from formatting import qualitative_palette, heating_palette
\end{pycode}

%%%%%%%%%%%%%%%%%%%%%%%%%%%%%%%%%%%%%% Field Extrapolation %%%%%%%%%%%%%%%%%%%%
\subsection{Magnetic Field Extrapolation}\label{field}

\begin{pycode}[manager_methods]
from sunpy.instr.aia import aiaprep
from sunpy.physics.differential_rotation import diffrot_map
####################################################
#                     Data Prep                    #
####################################################
aia_map = Map(manager_methods.data_file('aia_171_observed.fits'))
hmi_map = Map(manager_methods.data_file('hmi_magnetogram.fits'))
# AIA
aia_map = diffrot_map(aiaprep(aia_map), time=hmi_map.date, rot_type='snodgrass')
aia_map = aia_map.submap(
    SkyCoord(-440, -375, unit=u.arcsec, frame=aia_map.coordinate_frame),
    SkyCoord(-140, -75, unit=u.arcsec, frame=aia_map.coordinate_frame),
)
# HMI
hmi_map = hmi_map.rotate(order=3)
hmi_map = aiaprep(hmi_map).submap(
    aia_map.bottom_left_coord, aia_map.top_right_coord)
####################################################
#                       Plot                       #
####################################################
fig = plt.figure(figsize=texfigure.figsize(pytex, scale=1.0, height_ratio=0.5,       
                                            figure_width_context='textwidth'))
plt.subplots_adjust(wspace=0.03)
### HMI ###
ax = fig.add_subplot(121, projection=hmi_map)
hmi_map.plot(
    title=False,annotate=False,
    norm=matplotlib.colors.SymLogNorm(50, vmin=-7.5e2, vmax=7.5e2),
    cmap=matplotlib.colors.LinearSegmentedColormap.from_list('', ['C0','w','C3'])
)
ax.grid(alpha=0)
# HPC Axes
lon,lat = ax.coords[0],ax.coords[1]
lat.set_ticklabel(fontsize=plt.rcParams['ytick.labelsize'])
lon.set_ticklabel(fontsize=plt.rcParams['xtick.labelsize'])
lon.set_axislabel(r'Helioprojective Longitude [arcsec]', fontsize=plt.rcParams['axes.labelsize'])
lat.set_axislabel(r'Helioprojective Latitude [arcsec]', fontsize=plt.rcParams['axes.labelsize'])
# HGS Axes
hgs_lon,hgs_lat = aia_map.draw_grid(axes=ax,grid_spacing=10*u.deg,alpha=0.5,color='k')
hgs_lat.set_axislabel_visibility_rule('labels')
hgs_lon.set_axislabel_visibility_rule('labels')
hgs_lat.set_ticklabel_visible(False)
hgs_lon.set_ticklabel_visible(False)
hgs_lat.set_ticks_visible(False)
hgs_lon.set_ticks_visible(False)
### AIA ###
ax = fig.add_subplot(122, projection=aia_map,)
# Plot image
aia_map.plot(
    title=False,annotate=False,
    norm=ImageNormalize(vmin=0,vmax=5e3,stretch=AsinhStretch(0.1)))
# Plot fieldlines
ar = synthesizAR.Field.restore(os.path.join(manager_methods.data_dir, 'base_noaa1158'), lazy=True)
for l in ar.loops[::10]:
    c = l.coordinates.transform_to(aia_map.coordinate_frame)
    ax.plot_coord(c, '-', color='w', lw=0.5, alpha=0.25)
ax.grid(alpha=0)
# HMI Contours
hmi_map.draw_contours(u.Quantity([-5,5], '%'), axes=ax, colors=['C0', 'C3'], linewidths=0.75)
# HPC Axes
lon,lat = ax.coords[0],ax.coords[1]
lon.set_ticks(color='w')
lat.set_ticks(color='w')
lon.set_ticklabel(fontsize=plt.rcParams['xtick.labelsize'])
lat.set_ticklabel_visible(False)
lon.set_axislabel('')
lat.set_axislabel_visibility_rule('labels')
# HGS Axes
hgs_lon,hgs_lat = aia_map.draw_grid(axes=ax,grid_spacing=10*u.deg,alpha=0.5,color='w')
hgs_lat.set_axislabel_visibility_rule('labels')
hgs_lon.set_axislabel_visibility_rule('labels')
hgs_lat.set_ticklabel_visible(False)
hgs_lon.set_ticklabel_visible(False)
hgs_lat.set_ticks_visible(False)
hgs_lon.set_ticks_visible(False)
####################################################
#                  Save figure                     #
####################################################
fig_aia_hmi_lines = manager_methods.save_figure('magnetogram',)
fig_aia_hmi_lines.caption = r'Line-of-sight magnetic field strength as observed by HMI (left) and AIA 171 \AA{} observation (right) of \AR{} NOAA 1158 on 12 February 2011 15:33 UTC. The gridlines show the heliographic longitude and latitude. In the right panel, 500 out of the total 5000 fieldlines are overlaid and the red and blue contours show the HMI observation at the $+5\%$ (red) and $-5\%$ (blue) levels.'
fig_aia_hmi_lines.figure_env_name = 'figure*'
fig_aia_hmi_lines.figure_width = r'\textwidth'
fig_aia_hmi_lines.placement = ''
fig_aia_hmi_lines.fig_str = fig_str
\end{pycode}
\py[manager_methods]|fig_aia_hmi_lines|

We choose \AR{} NOAA 1158, as observed by the Helioseismic Magnetic Imager \citep[HMI,][]{hoeksema_helioseismic_2014} on 12 February 2011 15:33 UTC, from the list of active regions studied by \citet{warren_systematic_2012}. The line-of-sight magnetogram is shown in the left panel of \autoref{fig:magnetogram}. We model the geometry of \AR{} NOAA 1158 by computing the three-dimensional magnetic field using the oblique potential field extrapolation method of \citet{schmidt_observable_1964} as outlined in \citet[section 3]{sakurai_greens_1982}. The extrapolation technique of \citeauthor{schmidt_observable_1964} is well-suited for our purposes due to its simplicity and efficiency though we note it is only applicable on the scale of an \AR{}. We include the oblique correction to account for the fact that the \AR{} is off of disk center.

\begin{pycode}[manager_methods]
fig = plt.figure(figsize=texfigure.figsize(pytex, scale=1.0, height_ratio=1.0,
                                            figure_width_context='columnwidth'))
ax = fig.gca()
vals,bins,_ = ax.hist(
    [l.full_length.to(u.Mm).value for l in ar.loops[::10]],
    bins='scott', color='k', histtype='step', lw=plt.rcParams['lines.linewidth'])
ax.set_xlabel(r'$L$ [Mm]');
ax.set_ylabel(r'Number of Loops');
ax.set_ylim(-100,1300)
ax.set_xlim(-1,260)
# Spines
ax.spines['top'].set_visible(False)
ax.spines['right'].set_visible(False)
ax.spines['left'].set_bounds(ax.get_yticks()[1], ax.get_yticks()[-2])
ax.spines['bottom'].set_bounds(ax.get_xticks()[1], ax.get_xticks()[-2])
fig_loop_dist = manager_methods.save_figure('loops',)
fig_loop_dist.caption = r'Distribution of loop lengths (in Mm) traced from field extrapolation of magnetogram of NOAA 1158. A total of 5000 fieldlines were traced through the extrapolated volume.'
fig_loop_dist.figure_width = r'\columnwidth'
fig_loop_dist.placement = ''
fig_loop_dist.fig_str = fig_str
\end{pycode}
\py[manager_methods]|fig_loop_dist|

After computing the three-dimensional vector field from the observed magnetogram, we trace $5\times10^3$ fieldlines through the extrapolated volume using the streamline tracing functionality in the yt software package \citep{turk_yt_2011}. We choose only closed fieldlines in the range $20<L<300$ Mm, where $L$ is the full length of the loop. The right panel of \autoref{fig:magnetogram} shows a subset of the traced loops overlaid on the observed AIA 171 \AA{} image of NOAA 1158. Contours from the observed HMI LOS magnetogram are overlaid in red(positive) and blue (negative). A qualitative comparison between the extrapolated fieldlines and the loops visible in the AIA 171 \AA{} image reveals that our field extrapolation and line tracing adequately captures the three-dimensional geometry of the \AR{}. \autoref{fig:loops} shows the distribution of full lengths for the traced loops. Note that a large portion of the loops are short loops located near the center of the \AR{}.

%%%%%%%%%%%%%%%%%%%%%%%%%%%%%%%%%%%%% Loop Hydrodynamics %%%%%%%%%%%%%%%%%%%%%%
\subsection{Loop Hydrodynamics}\label{loops}

Due to the low-$\beta$ nature of the corona, we can treat each loop traced from our field extrapolation as an isolated atmosphere. We use the enthalpy-based thermal evolution of loops (EBTEL) model \citep{klimchuk_highly_2008,cargill_enthalpy-based_2012}, specifically the two-fluid version of the EBTEL model \citep{barnes_inference_2016} to model the dynamics in each loop. The two-fluid EBTEL code solves the time-dependent, two-fluid hydrodynamic equations spatially-integrated over the coronal part of the loop for the electron pressure and temperature, ion pressure and temperature, and density. The two-fluid EBTEL model accounts for radiative losses in both the transition region and corona, thermal conduction (including flux limiting), and binary Coulomb collisions between electrons and ions. The time-dependent heating input is configurable and can be deposited in the electrons and/or ions. A detailed description of the model and a complete derivation of the two-fluid EBTEL equations can be found in Appendix B of \citet{barnes_inference_2016}.

For each of the $5\times10^3$ loops, we run a separate instance of the two-fluid EBTEL code for $3\times10^4$ s of simulation time to model the time-dependent, spatially-averaged coronal temperature and density. For each simulation, the loop length is determined from the field extrapolation. We include flux limiting in the heat flux calculation and use a flux limiter constant of 1 \citep[see Eqs. 21 and 22 of][]{klimchuk_highly_2008}. Additionally, we choose to deposit all of the heating into the electrons. Though EBTEL only computes spatially-averaged quantities in the coronal portion of the loop, its efficiency allows us to calculate time-dependent solutions for many thousands of loops in a few minutes. 

%%%%%%%%%%%%%%%%%%%%%%%%%%%%%%%%%%%%% Heating %%%%%%%%%%%%%%%%%%%%%%%%%%%%%%%%%
\subsection{Heating Model}\label{heating}

We parameterize the heating input in terms of discrete heating pulses on a single strand with triangular profiles of duration $\tau_{event}=200$ s. For each event $i$, there are two parameters: the peak heating rate $q_i$ and the waiting time prior to the event $t_{wait,i}$. We choose $q_i$ from a power-law distribution with slope $-2.5$ and we define the waiting time such that $t_{wait,i}$ is the amount of time between when event $i-1$ ends and event $i$ begins. Following the approach of \citet{cargill_active_2014}, we relate the waiting time and the event energy such that $t_{wait,i}\propto q_i$. The physical motivation for this scaling is as follows. In the nanoflare model of \citet{parker_nanoflares_1988}, random convective motions continually stress loops rooted in the photosphere, leading to the buildup and and eventual release of energy. If the field is stressed for a long amount of time without relaxation, large discontinuities will have time to develop in the field, leading to a dramatic release of energy. Conversely, if the field relaxes quickly, there is not enough time for the field to become sufficiently stressed and the resulting energy release will be relatively small. 

In this work we explore three different heating scenarios: low-, intermediate-, and high-frequency nanoflares. We define the \textit{heating frequency} in terms of the ratio between the loop cooling time, $\tau_{cool}$, and the average waiting time of all events on a given strand, $\langle t_{wait}\rangle$,

\begin{equation}\label{eq:heating_types}
    \varepsilon = \frac{\langle t_{wait}\rangle}{\tau_{cool}}
    \begin{cases} 
        < 1, &  \text{high frequency},\\
        \sim1, & \text{intermediate frequency}, \\
        > 1, & \text{low frequency}.
     \end{cases}
\end{equation}

We choose to parameterize the heating in terms of the cooling time rather than an absolute waiting time as $\tau_{cool}\sim L$ \citep[see appendix]{cargill_active_2014}. While a waiting time of 2000 s might correspond to low frequency heating for a 20 Mm loop, it would correspond to high frequency heating in the case of a 150 Mm loop. By parameterizing the heating in this way, we ensure that all loops in the \AR{} are heated at the same frequency relative to their cooling time. \autoref{fig:hydro-profiles} shows the heating rate, electron temperature, and density as a function of time, for a single loop, for the three heating scenarios listed above. 

\begin{pycode}[manager_methods]
fig,axes = plt.subplots(3, 1, sharex=True,
                        figsize=texfigure.figsize(pytex, scale=1.0, height_ratio=1.0,
                                                  figure_width_context='columnwidth'))
plt.subplots_adjust(hspace=0.)
colors = heating_palette()
i_loop=680
heating = ['high_frequency', 'intermediate_frequency','low_frequency']
loop = ar.loops[i_loop]
for i,h in enumerate(heating):
    loop.parameters_savefile = os.path.join(manager_methods.data_dir, f'{h}', 'loop_parameters.h5')
    with h5py.File(loop.parameters_savefile, 'r') as hf:
        q = np.array(hf[f'loop{i_loop:06d}']['heating_rate'])
    axes[0].plot(loop.time, q, color=colors[i], label=h.split('_')[0].capitalize(),)
    axes[1].plot(loop.time, loop.electron_temperature[:,0].to(u.MK), color=colors[i],)
    axes[2].plot(loop.time, loop.density[:,0]/1e9, color=colors[i],)
# Legend
axes[0].legend(ncol=3,loc="lower left", bbox_to_anchor=(0.,1.02),frameon=False,)
# Labels and limits
axes[0].set_xlim(0,3e4)
axes[0].set_yticks([0.005,0.015,0.025])
axes[1].set_ylim(0.1,8)
axes[1].set_yticks(axes[1].get_yticks()[1:])
axes[2].set_ylim(0,1.8)
#axes[2].set_yticks(axes[2].get_yticks()[::2])
axes[0].set_ylabel(r'$Q$ [erg$/$cm$^{3}$$/$s]')
axes[1].set_ylabel(r'$T$ [MK]')
axes[2].set_ylabel(r'$n$ [10$^9$ cm$^{-3}$]')
axes[2].set_xlabel(r'$t$ [s]')
# Spines
axes[0].spines['bottom'].set_visible(False)
axes[0].spines['top'].set_visible(False)
axes[0].spines['right'].set_visible(False)
axes[0].tick_params(axis='x',which='both',bottom=False)
axes[1].spines['top'].set_visible(False)
axes[1].spines['bottom'].set_visible(False)
axes[1].spines['right'].set_visible(False)
axes[1].tick_params(axis='x',which='both',bottom=False)
axes[2].spines['top'].set_visible(False)
axes[2].spines['right'].set_visible(False)
fig_hydro_profiles = manager_methods.save_figure('hydro-profiles')
fig_hydro_profiles.caption = r'Heating rate (top), electron temperature (middle), and density (bottom) as a function of time for the three heating scenarios for a single loop. The colors denote the heating frequency as defined in the legend. The corresponding loop has a half length of $L/2\approx40$ Mm and a mean field strength of $\bar{B}\approx30$ G.'
fig_hydro_profiles.figure_width = r'\columnwidth'
fig_hydro_profiles.placement = ''
fig_hydro_profiles.fig_str = fig_str
\end{pycode}
\py[manager_methods]|fig_hydro_profiles|

Additionally, for each heating frequency, we constrain the total flux into the \AR{} to be $F_{\ast}=10^7$ erg cm$^{-2}$ s$^{-1}$ \citep{withbroe_mass_1977}. For a single impulsive event $i$, the energy density is $E_i=\tau_{event} q_i/2$. Summing over all events on all loops that comprise the \AR{} gives,
\begin{equation}
    F_{AR} = \frac{\sum_l\sum_i E_iL_l}{t_{total}}
\end{equation}
where $t_{total}$ s is the total simulation time. To constrain $F_{AR}$ according to the observed total flux, we need to satisfy
\begin{equation}\label{eq:energy_constraint}
    \frac{| F_{AR}/N_{loops} - F_{\ast} |}{F_{\ast}} < \delta,
\end{equation}
where $\delta\ll1$ and $N_{loops}=5\times10^3$ is the total number of loops comprising the \AR{}. In order to satisfy \autoref{eq:energy_constraint}, we iteratively adjust the lower bound on the power-law distribution from which we choose $q_i$ until we have met the above condition within some numerical tolerance. Additionally, for each loop, we fix the upper bound of the event power-law distribution to be $\bar{B}_l^2/8\pi$ where $\bar{B}_l$ is the spatially-averaged field strength along the loop $l$ as derived from the field extrapolation. This is the maximum amount of energy that the field could possibly make available to the loop. Note that while the derived potential field is already in its lowest energy state and thus has no energy to give up, our goal here is only to understand how the distribution of field strengths may be related to the properties of the heating. In this way, we use the potential field as a proxy for the non-potential component of the coronal field, with the understanding that we cannot make any qualitative conclusions regarding the amount of available energy or the stability of the field itself.

\begin{deluxetable}{lcc}
    \tablecaption{All three heating models plus the two single-event control models. In the single-event models, the energy flux is not constrained by \autoref{eq:energy_constraint}.\label{tab:heating}}
    \tablehead{\colhead{Name} & \colhead{$\varepsilon$ (see Eq.\ref{eq:heating_types})} & \colhead{Energy Constrained?}}
    \startdata
    high frequency & 0.1 & yes \\
    intermediate frequency & 1 & yes \\
    low frequency & 5 & yes \\
    cooling & 1 event per loop & no \\
    random & 1 event per loop & no
    \enddata
\end{deluxetable}

In addition to these three multi-event heating models, we also run two single-event control models. In both control models every loop in the \AR{} is heated exactly once with an event of energy $\bar{B}_l^2/8\pi$. In our first control model, the start time of every event is $t=0$ s such that all loops are allowed to cool uninterrupted for $t_{total}=10^4$ s. In the second control model, the start time of the event on each loop is chosen from a uniform distribution over the interval $(0, 3\times10^4)$ s, ensuring that the heating is out of phase across all loops and thus maximally decoherent. In these two models, the energy has not been constrained according to \autoref{eq:energy_constraint} and the total flux into the \AR{} is $(\sum_{l}\bar{B}_l^2L_l)/8\pi t_{total}$. From here on, we will refer to these two models as the ``cooling'' and ``random'' models, respectively. All five heating scenarios are summarized in \autoref{tab:heating}.

%%%%%%%%%%%%%%%%%%%%%%%%%%%%%%%%%%%%%%% Forward Modeling %%%%%%%%%%%%%%%%%%%%%%
\subsection{Forward Modeling}\label{forward}

\subsubsection{Atomic Physics}\label{atomic}

For an optically thin, high-temperature, low-density plasma, the radiated power per unit volume, or \textit{emissivity}, of a transition $\lambda_{ij}$ of an electron in ion $k$ of element $X$ is given by,
\begin{equation}
    \label{eq:ppuv}
    P(\lambda_{ij}) = \frac{n_H}{n_e}\mathrm{Ab}(X)N_j(X,k)f_{X,k}A_{ji}\Delta E_{ji}n_e,
\end{equation}
where $N_j$ is the fractional energy level population of excited state $j$, $f_{X,k}$ is the fractional population of ion $k$, $\mathrm{Ab}(X)$ is the abundance of element $X$ relative to hydrogen, $n_H/n_e\approx0.83$ is the ratio of hydrogen and electron number densities, $A_{ji}$ is the Einstein coefficient, and $\Delta E_{ji}=hc/\lambda_{ij}$ is the energy of the emitted photon \citep[see][]{mason_spectroscopic_1994,del_zanna_solar_2018}. To compute \autoref{eq:ppuv}, we use version 8.0.6 of the CHIANTI atomic database \citep{dere_chianti_1997,young_chianti_2016}. We use the abundances of \citet{feldman_potential_1992} as provided by CHIANTI. For each atomic transition, $A_{ji}$ and $\lambda_{ji}$ can be looked up in the database. To find $N_j$, we solve the level-balance equations for ion $k$, including the relevant excitation and de-excitation processes as provided by CHIANTI \citep[see section 3.3 of][]{del_zanna_solar_2018}.

The ion population fractions, $f_{X,k}$, provided by CHIANTI assume \textit{ionization equilibrium} (i.e. the ionization and recombination rates are always in balance). However, in the rarefied solar corona, where the plasma is likely heated impulsively, it is not gauranteed that the ionization timescale is less than the heating timescale, meaning that the ionization state may not be representative of the electron temperature \citep{bradshaw_explosive_2006,reale_nonequilibrium_2008,bradshaw_numerical_2009}. To properly account for this effect, we compute $f_{X,k}$ by solving the time-dependent level population equations for each element using the ionization and recombination rates provided by CHIANTI. The details of this calculation are provided in \autoref{nei}.

\subsubsection{Instrument Effects}\label{instrument}

\begin{pycode}[manager_methods]
import io
from astropy.table import Table
from astropy.io import ascii
em = EmissionModel.restore(os.path.join(manager_methods.data_dir, 'base_emission_model.json'))
data = {'Element': [], 'Number of Ions': [], 'Number of Transitions': [],}
for i in em:
    if not hasattr(i.transitions, 'wavelength'):
        continue
    data['Element'].append(i.atomic_symbol)
    data['Number of Ions'].append(1)
    data['Number of Transitions'].append(i.transitions.wavelength.shape[0])
df = pd.DataFrame(data=data).groupby('Element').sum().reset_index()
z = df['Element'].map(plasmapy.atomic.atomic_number)
df = df.assign(z = z).sort_values(by='z', axis=0).drop(columns='z')
caption = r"Elements included in the calculation of \autoref{eq:intensity}. For each element, we include all ions for which CHIANTI provides sufficient data for computing the emissivity.\label{tab:elements}"
with io.StringIO() as f:
    ascii.write(Table.from_pandas(df), format='aastex', caption=caption, output=f)
    table = f.getvalue()
\end{pycode}
\py[manager_methods]|table|

To model the intensity as it would be observed by AIA, we combine \autoref{eq:ppuv} with the wavelength response function of the instrument and integrate along the line of sight (LOS),
\begin{equation}\label{eq:intensity}
    I_c = \frac{1}{4\pi}\sum_{\{ij\}}\int_{\text{LOS}}\mathrm{d}hP(\lambda_{ij})R_c(\lambda_{ij})
\end{equation}
where $I_c$ is the intensity for a given pixel in channel $c$, $P(\lambda_{ij})$ is the emissivity as given by \autoref{eq:ppuv}, $R_c$ is the wavelength response function of the instrument for channel $c$ \citep[see][]{boerner_initial_2012}, $\{ij\}$ is the set of all atomic transitions listed in \autoref{tab:elements}, and the integration is along the LOS. Note that when computing the intensity in each channel of AIA, we do not rely on the temperature response functions computed by SolarSoft and instead use the wavelength response functions directly. This is because the response functions returned by \texttt{aia\_get\_response.pro} assume both ionization equilibrium and constant pressure. \autoref{effective_response_functions} provides further details on our motivation to not use the precomputed temperature response functions.

Once we have computed the emissivity according to \autoref{eq:ppuv} for all of the transitions in \autoref{tab:elements} and for all the loops in our \AR{}, we compute the LOS integral in \autoref{eq:intensity} by first converting all of the loop coordinates to a Helioprojective (HPC) coordinate frame \citep[see][]{thompson_coordinate_2006}. To do this, we use the coordinate transformation functionality in Astropy \citep{the_astropy_collaboration_astropy_2018} combined with the solar coordinate frames provided by SunPy \citep{sunpy_community_sunpypython_2015}. This enables us to easily project our synthetic image along any arbitrary LOS simply by changing the location of the observer that defines the HPC frame. Here, our HPC frame is defined by an observer at the position of the SDO spacecraft at 12 February 2011 15:33 UTC (i.e. the time of the HMI observation of NOAA 1158 shown in \autoref{fig:magnetogram}).

We then compute a weighted two-dimensional histogram of these transformed coordinates, using the emissivity values at each coordinate as the weights. We construct the histogram such that the bin widths are consistent with the spatial resolution of the instrument. For AIA, a single bin, representing a single pixel, has a width of 0.6 arcseconds. Finally, we  apply a gaussian filter to the resulting histogram to simulate the point spread function of the instrument. We do this for each timestep, as defined by the cadence of the instrument, and for each channel. For each heating scenario, this gives us approximately $1.8\times10^4$ separate images.

%%%%%%%%%%%%%%%%%%%%%%%%%%%%%%%%%%%%%%%%%%%%%%%%%%%%%%%%%%%%%%%%%%%%%%%%%%%%%%%
%                                   Results                                   %
%%%%%%%%%%%%%%%%%%%%%%%%%%%%%%%%%%%%%%%%%%%%%%%%%%%%%%%%%%%%%%%%%%%%%%%%%%%%%%%
\section{Results}\label{results}
\authorcomment1{High, intermediate, and low frequency cases plus control models. How should these be divided up?}

Two questions we want to try to answer:
\begin{enumerate}
    \item How is the timelag related to the heating frequency?
    \item What heating model is most consistent with observed timelags?
\end{enumerate}

%%%%%%%%%%%%%%%%%%%%%%%%%%%%%%%%%%%%%%% Intensities %%%%%%%%%%%%%%%%%%%%%%%%%%%
\subsection{Intensities}

We compute the intensities for the 94, 131, 171, 193, 211, and 335 \AA{} channels of SDO/AIA using the procedure described in \autoref{forward}. We compute the intensity in each pixel of the model \AR{} over a total simulation period of $3\times10^4$ s$\approx8.3$ hours with the exception of the cooling case which is only run for $10^4$ s. We complete this procedure for each of the five heating scenarios described in \autoref{heating}. 

\autoref{fig:intensity_map} shows a snapshot of the intensity in each pixel of the \AR{} for each AIA channel at $t=15\times10^3$ for the three nanoflare heating cases.

\begin{figure*}
    \plotone{../figures/intensity_maps.pdf}
    \caption{Snapshots of intensity across the whole \AR{} at $t=15\times10^3$ s. The rows correspond to the six EUV channels of AIA and the columns are the three different heating frequencies. In each column, the colarbar is on a square root scale and is normalized between zero and the maximum intensity in the low-frequency case. The color tables are the standard AIA color tables as implemented in SunPy \citep{sunpy_community_sunpypython_2015}.}
    \label{fig:intensity_map}
\end{figure*}

%%%%%%%%%%%%%%%%%%%%%%%%%%%%%%%%%%%%%%% Timelag Maps %%%%%%%%%%%%%%%%%%%%%%%%%%%%%%
\subsection{Timelag Maps}\label{timelag_maps}

We apply the timelag method of \citet{viall_evidence_2012} to our simulated intensities for all of the heating scenarios discussed in \autoref{heating}. The details of the timelag calculation are given in \autoref{timelags}. For each pixel in the active region, we compute the cross-correlation for every possible channel pair (15 in total) and find the temporal offset which maximizes the correlation. This offset, or \textit{timelag}, given by \autoref{eq:timelag}, indicates how much the second light curve must be shifted in time compared to the first in order to maximize the cross-correlation between the two light curves. By convention, we order the channel pairs such that the hot channel is listed first, meaning that a positive timelag corresponds to the variability in the hotter channel followed by variability in the cooler channel, thus indicating cooling plasma. If the correlation in a given pixel is too low ($\max{\mathcal{C}_{AB}}<0.1$), we mask that pixel and color it white. 

\begin{figure*}
    \plotone{../figures/model_timelags.pdf}
    \caption{Timelag maps for three different channel pairs for all five of the heating models described in \autoref{heating}. The value of each pixel indicates the temporal offset which maximizes the cross-correlation (see \autoref{eq:timelag}). The columns indicate the different channel pairs and the rows indicate the three heating scenarios plus our two control cases. The colorbar ranges from -5000 s to +5000 s.}
    \label{fig:model_timelags}
\end{figure*}

\autoref{fig:model_timelags} shows $\tau_{AB}$ in each pixel of our simulated \AR{} for all heating scenarios listed in \autoref{tab:heating} and three selected channel pairs: 94-335 \AA{}, 211-131 \AA{}, and 193-1171 \AA{}. Blues correspond to negative timelags while reds correspond to positive timelags and pale yellow indicates zero timelag. The range of the colorbar is $\pm5000$ s. Note that the frequency decreases as we move left to right across each row.

We look first at the 94-335 \AA{} channel pair. For every heating scenario, a similar pattern emerges: near zero timelags in the inner core of the \AR{}, short positive timelags ($\approx2000$ s) in the outer core, and negative timelags on the outer edge of the \AR{}. Note that a positive timelag in this channel means that variability in the 335 channel follows variability in the 94 channel, implying that plasma is cooling through $\approx8$ MK down to $\approx2$ MK (see \autoref{fig:aia_response}). The increasing magnitude of the positive timelags as we move away from the center of the \AR{} is due to the distribution of loop lengths. Shorter loops are concentrated in the center of the \AR{} (see \autoref{fig:magnetogram}) and because $\tau_{cool}\propto L$ \citep[see Appendix of][]{cargill_active_2014}, longer loops take more time to cool through the same temperature interval. The negative timelags are due to the fact that these loops far from the center of the \AR{} are rooted in areas of weaker magnetic field and are not heated into the temperature range of the hot peak of the 94 \AA{} channel. Thus, the cooling from 335 \AA{} to the cooler peak of 94 \AA{} dominates the cross-correlation curve, producing a negative timelag. These negative timelags occur closer to the core as the heating frequency increases (from right to left in \autoref{fig:model_timelags}) because the more often a loop is heated, the weaker the individual events will be. 

In the 211-131 \AA{} channel pair, we see that as the frequency decreases, long positive timelags tend to dominate at the outer edge of the \AR{}, with the length of the timelag decreasing as we move toward the center of the \AR{}. As discussed above, this is due to the dependence of the loop cooling time on the loop length. This pattern is most apparent in the low frequency, random, and cooling cases (the last three columns of \autoref{fig:model_timelags}) while the high and intermediate frequency cases show a mix of positive and negative timelags across the whole \AR{} with few areas of spatially coherent positive timelags. At high and intermediate frequencies, the majority of the loops are reheated often enough that they are not allowed fully cool through the 131 \AA{}, meaning that the cooling behavior from 211 \AA{} to 131 \AA{} does not dominate the cross-correlation curve. Note that the positive timelags in the 211-131 \AA{} channel pair are significantly longer than those in the 94-335 \AA{} despite the temperature separation in first pair being significantly larger than the second pair, $\Delta T_{94-335}>\Delta T_{211-131}$. In the temperature range $2.5<T<7.3$ MK, the dominant loop cooling mechanism is thermal conduction while radiative cooling dominates in the range $0.6<T<2.5$ MK. Because thermal conduction is far more efficient, a loop spends less time in the $[T_{335},T_{94}]$ temperature range than in $[T_{131},T_{211}]$ despite the former being a wider interval.

Additionally, we note the presence of negative 211-131 \AA{} timelags in the center of the \AR{} in the low, intermediate, and high frequency cases. These negative timelags correspond to plasma cooling from the hot part of the 131 \AA{} channel through the 211 \AA{} channel and thus indicate  $\ge10$ MK plasma. Notably, these negative timelags are independent of heating frequency. Because these inner core loops are rooted in areas of strong field, enough energy is made available by the field (see discussion in \autoref{heating}) to heat  them well into the hot part of the 131 \AA{} passband. Furthermore, because these are relatively short loops, the density increases rapidly enough for this hot plasma to be visible before it is washed out by thermal conduction. Though we have not shown them here, similar negative timelag signatures are present in nearly all of the other 131 \AA{} channel pairs as well. These negative timelags are not present in our two control cases as the amount of energy supplied to each loop is less than that of the other three cases (see \autoref{tab:heating}).

Lastly, we look at results from the 193-171 \AA{} channel pair as shown in the last row of \autoref{fig:model_timelags}. Note that in all five heating scenarios, zero timelag dominates the inner core of the \AR{}. This underscores the point that zero timelags do not correspond to steady heating \citep[see][]{viall_transition_2015,viall_signatures_2016}. As in the other two channel pairs, we find positive timelags which decrease in duration as we move towards the center of the \AR{} and this pattern becomes more apparent as the frequency decreases. Furthermore, we find very few negative timelags other than in the high frequency case where the cross-correlation is less likely to have a ``preferred'' timelag. This is because, unlike the 94 \AA{} and 131 \AA{} channels, the 193 \AA{} and 171 \AA{} channels are strongly peaked about a single temperature. Thus, persistent, spatially-coherent negative timelags could only be caused by a heating signature from 0.9 MK up to 1.5 MK. 

\begin{figure*}
    \plotone{../figures/model_correlations.pdf}
    \caption{Same as \autoref{fig:model_timelags} except here we show the maximum value of $\mathcal{C}_{AB}$ in each pixel.}
    \label{fig:model_correlations}
\end{figure*}

\autoref{fig:model_correlations} shows the peak cross-correlation value, $\max\mathcal{C}_{AB}$, for each selected channel pair. Looking first at all three channel pairs, we see that, on average, the cross-correlation increases as the heating frequency decreases. Additionally, we find that the highest cross-correlations tend to be in the center of the \AR{} while the lowest tend to be on the outer edge. Furthermore, other than the ``cooling'' scenario, we find that there are large variations from one loop to the next for all heating frequencies.  

%%%%%%%%%%%%%%%%%%%%%%%%%%%%%%%%%%%%%%% Histograms of Timelags %%%%%%%%%%%%%%%%
\subsection{Histograms of Timelags}\label{timelag_histograms}

\authorcomment1{Compare different heating frequencies for different channel pairs using histograms; show how going from high to low frequencies relaxes to the cooling time between peak temperatures of channels}

\begin{figure*}
    \plotone{../figures/model_timelags_histograms.pdf}
    \caption{Histograms of timelag values across the whole \AR{}. The rows indicate the different channel pairs and the columns indicate the different heating models. Colors are used to denote the various heating models. The bin range is $\pm5000$ s and the bin width is 30 s. As with the timelag maps, we do not include timelags corresponding to $\mathcal{C}_{AB}<0.1$.}
    \label{fig:timelag_histograms}
\end{figure*}

\autoref{fig:timelag_histograms} shows histograms of timelags for every possible channel pair and all five heating scenarios. Each histogram is colored according the corresponding heating function. The timelags are binned between -5000 s and +5000 s in 30 s bins. The columns are arranged such that heating frequency decreases from left to right while coherence increases from left to right, with the ``cooling only'' case representing maximally coherent evolution over the whole \AR{}.

Looking at each channel pair, we make two immediate observations: 
\begin{enumerate}
\item As the heating frequency decreases, the number of negative timelags decreases for nearly every channel pair.
\item In over half of the channel pairs, the distribution of positive timelags narrows as the heating frequency decreases
\end{enumerate}

Regarding our first point, in the cooling case, the negative timelag distribution goes to zero for many of the channel pairs, the few exceptions being those pairs involving strongly doubly-peaked channels, i.e. 94 \AA{} and 131 \AA{}. In the case of these channels which have both a hot and a cool peak, we expect to find negative timelags, even in our maximally coherent cooling case, because our convention of ordering the ``hot'' channel first has been violated. Thus, cooling plasma can lead to negative timelags. For the remaining channel pairs, negative timelags are associated with the loop heating and cooling cycle being interrupted by repeated events on a given strand. 

Dealing with our second point, nearly all of those pairs that do not exhibit this behavior include either the doubly-peaked 94 \AA{} and/or 131 \AA{} channels. 

\authorcomment2{Figure showing timelag maps for all 5 heating models; need to select which channel pairs: 131 pair, 94 pair, what else? Say three channel pairs so we have a 5-by-3 panel figure, split 3-by-3, 2-by-3 across two pages.}

\authorcomment1{Give a thorough qualitative discussion about how the heating frequency is related to the timelags}

\authorcomment1{Relationship to properties of the magnetic field? Energy density?}

%%%%%%%%%%%%%%%%%%%%%%%%%%%%%%%%%%%%%%% Emission Measure Slopes %%%%%%%%%%%%%%%%
\subsection{Emission Measure Slopes}

In addition to the timelag, we also compute the emission measure distribution (\dem) using the method of \citet{hannah_differential_2012} using both our simulated and observed AIA data. \dem provides a measure of how much plasma is emitting over a range of temperatures and has been used by many workers to assess heating frequency in \AR{} cores \citep[][and references therein]{tripathi_emission_2011,warren_constraints_2011,warren_systematic_2012,schmelz_cold_2012,bradshaw_diagnosing_2012,reep_diagnosing_2013,barnes_inference_2016,barnes_inference_2016,barnes_inference_2016-1}. 

A common technique is to fit a power-law $\mathrm{EM}(T)\sim T^a$ to \dem between approximately 1 and 4 MK and is effectively a measure of how isothermal the distribution is. Many workers \citep[see Table 3 of][and references therein]{bradshaw_diagnosing_2012} have found $2<a<5$. \citet{cargill_active_2014} used a range of waiting times dependent on the event energy (see \autoref{heating}) and found that he could account for the observed range of slopes. Thus, the \dem slope $a$ is one important diagnostic for assessing how often a single loop may be reheated. We compute $a$ between $10^6$ and $10^{6.6}$ K for each pixel in our \AR{}.

\authorcomment3{This whole section will probably be cut and moved to a different paper}

%%%%%%%%%%%%%%%%%%%%%%%%%%%%%%%%%%%%%%% Observations %%%%%%%%%%%%%
\subsection{Observations}\label{observations}

\begin{figure*}
    \plotone{../figures/observed_timelags.pdf}
    \caption{Timelag maps as calculated from intensity data from observations of \AR{} NOAA 1158 by SDO/AIA. Timelag maps are shown for every possible channel pair as indicated in the upper left corner of each map. As in \autoref{fig:model_timelags}, the colorbar range from -5000 s to +5000 s.}
    \label{fig:observed_timelags}
\end{figure*}

\begin{figure*}
    \plotone{../figures/observed_correlations.pdf}
    \caption{Same as \autoref{fig:observed_timelags} except here we show the maximum value of the cross-correlation as derived from the observations.}
    \label{fig:observed_correlations}
\end{figure*}

\authorcomment2{Discussion of observed timelags and cross-correlation values}

\authorcomment2{Possibly put histograms of observations versus models here too}

\authorcomment1{Emphasize the point that no single heating model is consistent with the observations. Need multiple heating frequencies; this will lead into random forest stuff}

%%%%%%%%%%%%%%%%%%%%%%%%%%%%%%%%%%%%%%% Pixel Classification %%%%%%%%%%%%%
\subsection{Pixel Classification}\label{classify}

\authorcomment1{Describe random forest technique; how is data prepared; what is the RF actually doing}

\begin{figure*}
    \plotone{../figures/probability_maps.pdf}
    \caption{Foo bar}
    \label{fig:probability_maps}
\end{figure*}

\authorcomment1{Describe and interpret the results of the classification}

\begin{figure}
    \plotone{../figures/frequency_map.pdf}
    \caption{foo bar}
    \label{fig:frequency_map}
\end{figure}

%%%%%%%%%%%%%%%%%%%%%%%%%%%%%%%%%%%%%%%%%%%%%%%%%%%%%%%%%%%%%%%%%%%%%%%%%%%%%%%
%                                   Summary and Conclusions                   %
%%%%%%%%%%%%%%%%%%%%%%%%%%%%%%%%%%%%%%%%%%%%%%%%%%%%%%%%%%%%%%%%%%%%%%%%%%%%%%%

\section{Summary}\label{conclusions}

We have carried out a series of numerical simulations in an effort to understand how signatures of the nanoflare heating frequency are manifested in two observables: the emission measure slope and the time lag. Additionally, we described each component of our pipeline for forward modeling \AR{} emission. For a given magnetogram observation of the relevant \AR{} (in this case, NOAA 1158), we compute a potential field extrapolation and trace a large number of field lines through the extrapolated vector field. For each traced field line, we run an EBTEL hydrodynamic simulation and use the resulting temperatures and densities, combined with data from CHIANTI and the instrument response function, to compute the time-dependent intensity. These intensities are then mapped back to the magnetic skeleton and integrated along the LOS in each pixel to create time-dependent images of the \AR{}.

Using our novel and efficient forward modeling pipeline, we produced AIA images for all six EUV channels for $\approx8$ hours of simulation time. From these results, we computed both the emission measure slope and the time lag for all possible channel pairs. We carried out these steps for three different nanoflare heating frequencies, high, intermediate, and low, (see \autoref{eq:heating_types}) in addition to two control models, for a total of five different heating scenarios (see \autoref{tab:heating}).

Our results can be summarized in the following points:
\begin{enumerate}
    \item As the heating frequency decreases, the emission measure slope, $a$, becomes increasingly shallow, saturating at $a\approx2$. As the heating frequency increases, the distribution of slopes over the \AR{} is shifted to higher values and broadens.
    \item The time lag becomes increasingly spatially coherent with decreasing heating frequency. When strands are allowed to cool without being re-energized, the spatial distribution of time lags is largely determined by the distribution of loop lengths over the \AR{}.
    \item The distribution of time lags becomes increasingly broad as the heating frequency increases, consistent with the results of \citet{viall_signatures_2016}.
    \item Negative time lags in channel pairs where the second (``cool'') channel is 131 \AA{} provide a possible diagnostic for $\ge10$ MK plasma
\end{enumerate}

In this paper, we have used our advanced forward modeling pipeline to systematically examine how the emission measure slope and time lag are affected by the nanoflare heating frequency. In \citetalias{barnes_understanding_2019-1}, we use the model results presented here to train a random forest classifier and apply it to emission measure slopes and time lags derived from real AIA observations of NOAA 1158. The 15 channel pairs for the time lag and cross-correlation combined with the emission measure slope represent a 31-dimensional feature space and a single 500-by-500 pixel \AR{} amounts to $2.5\times10^5$ sample points. Performing an accurate assessment over this amount of data manually or ``by eye'' is at least impractical and likely impossible. Thus, the application of machine learning to the problem of assessing models in the context of real data is a critical step in understanding the underlying energy deposition in \AR{} cores and, to our knowledge, has not yet been applied in this context.  

%%%%%%%%%%%%%%%%%%%%%%%%%%%%%%%%%%%%%%%%%%%%%%%%%%%%%%%%%%%%%%%%%%%%%%%%%%%%%%%
%                                   Acknowledgment                            %
%%%%%%%%%%%%%%%%%%%%%%%%%%%%%%%%%%%%%%%%%%%%%%%%%%%%%%%%%%%%%%%%%%%%%%%%%%%%%%%
\acknowledgments
%%%%%%%%%%%%%%%%%%%%%%%%%%%%%%%%%%%%%%%%%%%%%%%%%%%%%%%%%%%%%%%%%%%%%%%%%%%%%%%
%                                   Software                                  %
%%%%%%%%%%%%%%%%%%%%%%%%%%%%%%%%%%%%%%%%%%%%%%%%%%%%%%%%%%%%%%%%%%%%%%%%%%%%%%%
\software{
    Astropy \citep{the_astropy_collaboration_astropy_2018},
    Dask\citep{dask_development_team_dask_2016},
    Matplotlib\citep{hunter_matplotlib_2007},
    NumPy\citep{oliphant_guide_2006},
    seaborn\citep{waskom_seaborn_2018},
    SunPy\citep{sunpy_community_sunpypython_2015},
    yt\citep{turk_yt_2011}
}
%%%%%%%%%%%%%%%%%%%%%%%%%%%%%%%%%%%%%%%%%%%%%%%%%%%%%%%%%%%%%%%%%%%%%%%%%%%%%%%
%                                   Appendix                                  %
%%%%%%%%%%%%%%%%%%%%%%%%%%%%%%%%%%%%%%%%%%%%%%%%%%%%%%%%%%%%%%%%%%%%%%%%%%%%%%%
\appendix
\begin{pycode}[manager_appendix]
manager_appendix = texfigure.Manager(
    pytex, './',
    number=3,
    python_dir='python',
    fig_dir='figures',
    data_dir='data',
)
from formatting import qualitative_palette, heating_palette
\end{pycode}
%%%%%%%%%%%%%%%%%%%%%%%%%%%%%%%%%%%%%%%%%%%%%%%%%%%%%%%%%%%%%%%%%%%%%%%%%%%%%%%
%                                   Appendix 1                                %
%%%%%%%%%%%%%%%%%%%%%%%%%%%%%%%%%%%%%%%%%%%%%%%%%%%%%%%%%%%%%%%%%%%%%%%%%%%%%%%
\section{Non-equilibrium Ion Populations}\label{nei}

In order to account for ionization non-equilibrium, we compute $f_{X,k}$ as a function of time $t$ by solving the time-dependent level population equations for each ion $k$ in each element $X$,
\begin{equation}\label{eq:nei}
    \frac{\partial f_k}{\partial t} = n_e(R_{k+1}f_{k+1} + I_{k-1}f_{k-1} - I_kf_k - R_kf_k)
\end{equation}
where $n_e$ is the electron density and $R_k$ and $I_k$ are the temperature-dependent recombination and ionization rates of ion $k$, respectively. The temperature-dependent ionization and recombination rates are computed using the data provided in CHIANTI. The ionization rates include both direct ionization and excitation autoionization and the recombination rates include both radiative and dielectronic recombination \citep[see section 6 of][]{young_chianti_2016}.

Note that for an element with atomic number $Z$, we must solve $Z+1$ coupled differential equations to find the non-equilibrium level populations. Setting the left hand side of \autoref{eq:nei} to zero gives the equation of ionization equilibrium. Casting \autoref{eq:nei} in matrix form,
\begin{equation}\label{eq:nei_mat}
    \dot{\mathbf{F}} = \mathbf{A}\mathbf{F},
\end{equation}
where $\mathbf{F}=(f_1,f_2,\ldots,f_k,\ldots,f_{Z+1})$ and $\mathbf{A}$ is a ${Z+1\times Z+1}$ tridiagonal matrix containing the ionization and recombination rates, multiplied by the electron density,

\begin{equation*}
    \mathbf{A} = n_e
        \begin{pmatrix}
            -(I_0 + R_0) & R_1 & 0 & \dots & 0 \\
            I_0 & -(I_1 + R_1) & R_2 & \dots & 0 \\
             & \ddots & \ddots & &  \\
            \vdots & I_{i-1} & -(I_i + R_i) & R_{i+1} & \vdots \\
             & & \ddots & \ddots & \\
            0 & \dots & I_{Z-2} & -(I_{Z-1} + R_{Z-1}) & R_Z \\
            0 & \dots & 0 & I_{Z-1} & -(I_Z + R_Z) 
        \end{pmatrix}.
\end{equation*}

Due to drastic changes in the ionization and recombination rates with temperature, the above system of equations is very ``stiff,'' making explicit schemes extremely sensitive to the choice of timestep \citep{macneice_numerical_1984,bradshaw_numerical_2009}. To solve \autoref{eq:nei_mat}, we use the ``deferred correction'' method of \citet{npl_modern_1961}, as pointed out by \citet{macneice_numerical_1984},
\begin{equation*}
    \mathbf{F}_{j+1} = \mathbf{F}_j + \frac{\Delta t}{2}(\dot{\mathbf{F}}_{j+1} + \dot{\mathbf{F}}_j) + \mathrm{h.o.t.}
\end{equation*}
where the index $j$ represents time. Rearranging the above expression and using \autoref{eq:nei_mat} yields an expression for $\mathbf{F}_{j+1}$,
\begin{equation}\label{eq:nei_iterative}
    \mathbf{F}_{j+1} \approx \left(\mathbb{I} - \frac{\Delta t}{2}\mathbf{A}_{j+1}\right)^{-1}\left(\mathbb{I} + \frac{\Delta t}{2}\mathbf{A}_{j}\right)\mathbf{F}_j
\end{equation}
where $\mathbb{I}$ is the identity matrix. To solve \autoref{eq:nei_mat}, we need only compute $\mathbf{A}_j$ for each $T(t_j)$, set $\mathbf{F}_0$ to the equilibrium ion populations, and iteratively compute \autoref{eq:nei_iterative} for all $j$. We solve \autoref{eq:nei_mat} for all elements in \autoref{tab:elements} and for each loop in the \AR{}.

%%%%%%%%%%%%%%%%%%%%%%%%%%%%%%%%%%%%%%%%%%%%%%%%%%%%%%%%%%%%%%%%%%%%%%%%%%%%%%%
%                                   Appendix 2                                %
%%%%%%%%%%%%%%%%%%%%%%%%%%%%%%%%%%%%%%%%%%%%%%%%%%%%%%%%%%%%%%%%%%%%%%%%%%%%%%%
\section{Effective AIA Response Functions}\label{effective_response_functions}

\begin{pycode}[manager_appendix]
from synthesizAR.instruments import InstrumentSDOAIA
from scipy.interpolate import splev
aia = InstrumentSDOAIA([0,1]*u.s, observer_coordinate=None)
em = EmissionModel.restore(os.path.join(manager_appendix.data_dir, 'base_emission_model.json'))
fig,axes = plt.subplots(2, 3, sharex=True, sharey=True,
                        figsize=texfigure.figsize(pytex, scale=1.0, height_ratio=2/3,       
                                                  figure_width_context='textwidth'))
T = np.logspace(5,8,100)
p = 10**(15)*u.K/(u.cm**3)
const_p_indices = np.array([(i,np.argmin(np.fabs(em.density.value-d.value))) 
                            for i,d in enumerate(p/em.temperature)])
with h5py.File(os.path.join(manager_appendix.data_dir, 'effective_aia_response.h5'), 'r') as hf:
    colors = qualitative_palette(len([k for k in hf['94']])-1)
    for i, (ax, channel) in enumerate(zip(axes.flatten(), aia.channels)):
        grp = hf[channel['name']]
        real_response = splev(T, channel['temperature_response_spline'])
        ax.plot(T, real_response, ls='-',color='k',)
        ax.plot(em.temperature, 
                np.array(grp['response'])[const_p_indices[:,0],const_p_indices[:,1]],
                color='k',ls='--',)
        elements = sorted([e for e in grp if e != 'response'],
                            key=lambda x: plasmapy.atomic.atomic_number(x))
        for j,element in enumerate(elements):
            ax.plot(em.temperature, 
                    np.array(grp[element])[const_p_indices[:,0],const_p_indices[:,1]],
                    color=colors[j], ls='--', label=plasmapy.atomic.atomic_symbol(element),)
        if i==0:
            ax.legend(ncol=len(elements), loc="lower left", bbox_to_anchor=(0.25,1.02),
                        frameon=False)
        ax.text(1e7,3e-25,'{} $\mathrm{{\AA}}$'.format(channel['name']),
                fontsize=plt.rcParams['axes.labelsize'])
ax.set_xscale('log')
ax.set_yscale('log')
ax.set_ylim([2e-30,2e-24])
ax.set_xlim([1e5,8e7])
ax.set_yticks([1e-29, 1e-27, 1e-25])
axes[1,0].set_ylabel(r'$K_c$ [DN cm$^5$ s$^{-1}$ pixel$^{-1}$]')
axes[1,0].set_xlabel(r'$T$ [K]')
plt.subplots_adjust(wspace=0.,hspace=0.)
fig_aia_response = manager_appendix.save_figure('aia-response')
fig_aia_response.caption = r'SolarSoft temperature response functions (solid black) and effective temperature response functions for the elements in \autoref{tab:elements} (dashed black) for all six EUV AIA channels. The colored, dashed curves, as indicated in the legend, denote the contributions of the individual elements to the total response. For this calculation, we have assumed equilibrium ionization. \authorcomment3{Possibly move this to an appendix}'
fig_aia_response.figure_env_name = 'figure*'
fig_aia_response.figure_width = r'\textwidth'
fig_aia_response.placement = ''
\end{pycode}
\py[manager_appendix]|fig_aia_response|

\autoref{fig:aia-response} shows the effective temperature response functions for the six EUV channels on AIA compared to those calculated from \texttt{aia\_get\_response.pro} in SolarSoft \citep{freeland_data_1998}. Even though we include a limited number of transitions from the CHIANTI database (see \autoref{tab:elements}), we recover nearly all of the response from each channel. The high-temperature contributions in the SolarSoft functions are due to continuum emission which we do not include in our model. In all cases, the continuum contribution is several orders of magnitude below peak of the channel response.

\begin{pycode}[manager_appendix]
fig = plt.figure(figsize=texfigure.figsize(pytex, scale=0.5, height_ratio=1.0, 
                                           figure_width_context='columnwidth'))
ax = fig.gca()
min_T = 1e300*u.K
max_T = 0*u.K
colors = heating_palette()
ar = synthesizAR.Field.restore(os.path.join(manager_appendix.data_dir, 'base_noaa1158'), lazy=True)
i_loop=680
loop = ar.loops[i_loop]
for i,h in enumerate(heating):
    loop.parameters_savefile = os.path.join(manager_appendix.data_dir, f'{h}', 'loop_parameters.h5')
    ax.plot(loop.electron_temperature[:,0], loop.density[:,0], color=colors[i], alpha=0.75,
            label=h.split('_')[0].capitalize())
    min_T = min(min_T, loop.electron_temperature.min())
    max_T = max(max_T, loop.electron_temperature.max())
p = 1e15*u.K/(u.cm**3)
T = np.linspace(min_T,max_T,1000)
ax.plot(T, p/T, color='k')
ax.set_xscale('log')
ax.set_yscale('log')
ax.set_xlim(1.75e5,1e7)
ax.set_ylim(1.01e7,6e9)
ax.legend(ncol=3,loc="lower left", bbox_to_anchor=(0.,1.02), frameon=False,)
ax.spines['top'].set_visible(False)
ax.spines['right'].set_visible(False)
ax.spines['bottom'].set_bounds(2e5,1e7)
ax.spines['left'].set_bounds(2e7,6e9)
ax.set_ylabel(r'$n$ [cm$^{-3}$]')
ax.set_xlabel(r'$T$ [K]')
fig_hydro_phase_space = manager_appendix.save_figure('hydro-phase-space')
fig_hydro_phase_space.caption = r'$n-T$ phase-space orbits for a single loop for the three heating scenarios as defined by the legend. The black line indicates a constant pressure of $10^{15}$ K cm$^{-3}$.'
fig_hydro_phase_space.figure_width = r'0.5\columnwidth'
\end{pycode}
\py[manager_appendix]|fig_hydro_phase_space|

As discussed in \autoref{atomic}, the assumption of ionization equilibrium is likely to be violated in the impulsive heating cases considered here. Thus, we must recompute the contributions of each ion, using the result of \autoref{eq:nei_mat} in place of the equilibrium ionization fractions. Furthermore, during the evolution of a loop, the pressure is not constant for any of our heating scenarios as evidenced by \autoref{fig:hydro-phase-space}. The black line of constant pressure $p=10^{15}$ K cm$^{-3}$ shows the default pressure at which the SolarSoft AIA response functions are evaluated. By recomputing and interpolating the emissivity, we ensure that we are evaluating all quantities in \autoref{eq:ppuv} at the correct temperature and density.

%%%%%%%%%%%%%%%%%%%%%%%%%%%%%%%%%%%%%%%%%%%%%%%%%%%%%%%%%%%%%%%%%%%%%%%%%%%%%%%
%                                   Appendix 3                                %
%%%%%%%%%%%%%%%%%%%%%%%%%%%%%%%%%%%%%%%%%%%%%%%%%%%%%%%%%%%%%%%%%%%%%%%%%%%%%%%
\section{Computing Timelags}\label{timelag_details}

To find the associated timelag for a channel pair in a given pixel, we compute the cross-correlation between the timeseries associated with each channel and find the delay which maximizes this cross-correlation. We can express the cross-correlation $\mathcal{C}$ between two channels $A$ and $B$ as,
\begin{equation}\label{eq:cc_pre}
    \mathcal{C}_{AB}(\tau) = \mathcal{I}_A(t)\star\mathcal{I}_B(t) = \mathcal{I}_A(-t)\ast\mathcal{I}_B(t)
\end{equation}
where $\star$ and $\ast$ represent the correlation and convolution operators, respectively, $\tau$ is the lag and
\begin{equation*}
    \mathcal{I}_c(t)=\frac{I_c(t)-\bar{I}_c}{\sigma_{c}},
\end{equation*}
is the mean-subtracted and scaled intensity of channel $c$ as a function of time. Taking the fourier transform of both sides of \autoref{eq:cc_pre} and using the convolution theorem,
\begin{align*}
    \mathcal{F}\{\mathcal{C}_{AB}(\tau)\} &= \mathcal{F}\{\mathcal{I}_A(-t)\ast\mathcal{I}_B(t)\},\\
    &= \mathcal{F}\{\mathcal{I}_A(-t)\}\mathcal{F}\{\mathcal{I}_B(t)\}.
\end{align*}
Taking the inverse Fourier transform, $\mathcal{F}^{-1}$, of both sides gives,
\begin{align}\label{eq:cc}
    \mathcal{C}_{AB}(\tau) &= \mathcal{F}^{-1}\{\mathcal{F}\{\mathcal{I}_A(-t)\}\mathcal{F}\{\mathcal{I}_B(t)\}\}.
\end{align}
Scaling \autoref{eq:cc} by the length of the intensity timeseries $I(t)$ yields the same result as that of the correlation defined in section 2 of \citet{viall_evidence_2012}. Furthermore, the \textit{timelag} between channels $A$ and $B$ is defined as,
\begin{equation}\label{eq:timelag}
    \tau_{AB} = \argmax_{\tau}\,\mathcal{C}_{AB}(\tau).
\end{equation}

By exploiting the definition of the cross-correlation as given in \autoref{eq:cc}, we can leverage existing Fourier transform algorithms in order to compute the $\mathcal{C}_{AB}$ in a scalable and efficient manner. For a 500-by-500 pixel active region observation and 15 possible channel pairs, we need to compute $\tau_{AB}$ nearly $4\times10^6$ times. We use the highly-optimized and thoroughly tested Fourier transform algorithms in the NumPy Python library for array computations \citep{oliphant_guide_2006} combined with the Dask library for parallel and distributed computing \citep{dask_development_team_dask_2016}. Using Dask, we are able to parallelize this operation over many cores such that, on a 64-core machine, we are able to compute timelags for all 15 channel pairs in every pixel of the \AR{} in less than ten minutes. For comparison, doing the same calculation by calling the IDL function \texttt{c\_correlate.pro} on each pixel in serial would take approximately fourteen hours for all 15 channel pairs.

\begin{pycode}[manager_appendix]
fig,axes = plt.subplots(
    2, 1, figsize=texfigure.figsize(pytex, scale=1.0,
                                    figure_width_context='columnwidth'))
plt.subplots_adjust(hspace=0.35)
colors = qualitative_palette(len(channels))
filename = os.path.join(manager_appendix.data_dir, 'cooling', 'timelag_1d_example.h5')
### Timeseries ###
with h5py.File(filename, 'r') as h5:
    grp = h5['timeseries']
    time = u.Quantity(grp['time'], grp['time'].attrs['unit'])
    for i,c in enumerate(channels):
        ts = u.Quantity(grp[f'{c}'], grp[f'{c}'].attrs['unit'])
        ts = ts/ts.max()
        axes[0].plot(time, ts, label='{} $\mathrm{{\AA}}$'.format(c),
                        color=colors[i])
axes[0].legend(frameon=False,ncol=1,loc=1)
axes[0].set_ylim(-0.05,1.01)
axes[0].set_xlim(-500,time[-1].to(u.s).value*1.001)
axes[0].set_yticks(np.linspace(0,1,3))
axes[0].set_xlabel(r'$t$ [s]')
axes[0].set_ylabel(r'$I_{c}/I_{c,max}$')
axes[0].spines['top'].set_visible(False)
axes[0].spines['right'].set_visible(False)
axes[0].spines['bottom'].set_bounds(axes[0].get_xticks()[1], axes[0].get_xticks()[-1])
axes[0].spines['left'].set_bounds(axes[0].get_yticks()[0], axes[0].get_yticks()[-1])
### Cross-correlations ###
with h5py.File(filename, 'r') as h5:
    grp = h5['cross_correlations']
    timelags = u.Quantity(grp['timelags'], grp['timelags'].attrs['unit'])
    for i,(ca,cb) in enumerate(selected_channel_pairs):
        cc = u.Quantity(grp[f'{ca}_{cb}'], grp[f'{ca}_{cb}'].attrs['unit'])
        axes[1].plot(timelags, cc, label='{}-{} $\mathrm{{\AA}}$'.format(ca,cb),
                        color=colors[i])
        axes[1].plot(timelags[cc.argmax()], cc.max(), marker='o', markersize=5,
                        color=colors[i], ls='')
axes[1].axvline(x=0,ls=':',color='k')
axes[1].set_xlim(-1.2*(5e3*u.s).value, 1.2*(5e3*u.s).value)
axes[1].set_ylim(-0.3, 1.01)
axes[1].set_yticks(np.linspace(-0.2,1,4))
axes[1].set_xticks(np.linspace(-5e3,5e3,6))
axes[1].legend(frameon=False,ncol=1,loc=2)
axes[1].set_ylabel(r'$\mathcal{C}_{AB}$')
axes[1].set_xlabel(r'$\tau$ [s]')
axes[1].spines['top'].set_visible(False)
axes[1].spines['right'].set_visible(False)
axes[1].spines['bottom'].set_bounds(axes[1].get_xticks()[0], axes[1].get_xticks()[-1])
axes[1].spines['left'].set_bounds(axes[1].get_yticks()[0], axes[1].get_yticks()[-1])
fig_timelag_example = manager_appendix.save_figure('timelag-example',)
fig_timelag_example.caption = r'Normalized pixel-averaged intensity timeseries for all six AIA EUV channels (top) and cross-correlation curves for selected channel pairs (bottom) for the ``cooling'' case. The dots indicate the peak in the cross-correlation curve.'
fig_timelag_example.figure_width = r'\columnwidth'
\end{pycode}
\py[manager_appendix]|fig_timelag_example|
    
\autoref{fig:timelag-example} shows an example calculation of \autoref{eq:cc} for a pixel-averaged timeseries from the ``cooling'' case described in \autoref{heating}. At each timestep, the intensity was averaged over a $5\arcsec\times5\arcsec$ patch centered on $(\theta_x,\theta_y)=(312.5\arcsec,332.5\arcsec)$. The top panel shows the intensity, normalized to the peak value, of each AIA channel while the bottom panel shows the cross-correlation curve, as a function of the offset, $\tau$, for three sample channel pairs. Note that the intensities peak in successively cooler channels as the loop plasma cools following the single heating event at $t=0$ s. The cross-correlation curves in the lower panel have peaks (denoted by dots) at lags ($\tau$) approximately equal to the separation of the intensity peaks in time. Note that all of the timelags are positive for these channel pairs though the $\mathcal{C}_{94-335}$ curve does show a weaker peak at a negative timelag due to the doubly-peaked nature of the 94 \AA{} response curve, with the cooler peak at a lower temperature than the 335 \AA{} channel (see \autoref{fig:aia-response}).

%%%%%%%%%%%%%%%%%%%%%%%%%%%%%%%%%%%%%%%%%%%%%%%%%%%%%%%%%%%%%%%%%%%%%%%%%%%%%%%
%                                   References                                %
%%%%%%%%%%%%%%%%%%%%%%%%%%%%%%%%%%%%%%%%%%%%%%%%%%%%%%%%%%%%%%%%%%%%%%%%%%%%%%%
\bibliographystyle{aasjournal.bst}
\bibliography{references.bib}
\end{document}
