%%%%%%%%%%%%%%%%%%%%%%%%%%%%%%%%%%%%%%%%%%%%%%%%%%%%%%%%%%%%%%%%%%%%%%%%%%%%%%%
%                                   Analysis                                  %
%%%%%%%%%%%%%%%%%%%%%%%%%%%%%%%%%%%%%%%%%%%%%%%%%%%%%%%%%%%%%%%%%%%%%%%%%%%%%%%
\section{Analysis}\label{analysis}

\authorcomment3{Should this whole section just go in the results section?}

\subsection{Timelag Analysis}\label{timelag_analysis}

We apply the timelag method of \citet{viall_evidence_2012} to both our simulated and observed intensities. To find the associated timelag for a channel pair in a given pixel, we compute the cross-correlation between the timeseries associated with each channel and find the lag which maximizes this cross-correlation. We can express the cross-correlation $\mathcal{C}$ between two channels $A$ and $B$ as,
\begin{equation}\label{eq:cc_pre}
    \mathcal{C}_{AB}(\tau) = \mathcal{I}_A(t)\star\mathcal{I}_B(t) = \mathcal{I}_A(-t)\ast\mathcal{I}_B(t)
\end{equation}
where $\star$ and $\ast$ represent the correlation and convolution operators, respectively, $\mathcal{I}_c(t)=(I_c(t)-\bar{I}_c)/\sigma_{I_c}$ is the mean-subtracted and scaled intensity of channel $c$ as a function of time, and $\tau$ is the lag. Taking the fourier transform of both sides of \autoref{eq:cc_pre} and using the convolution theorem,
\begin{align}\label{eq:cc}
    \mathcal{F}(\mathcal{C}_{AB}) &= \mathcal{F}(\mathcal{I}_A(-t)\ast\mathcal{I}_B(t)), \nonumber\\
    &= \mathcal{F}(\mathcal{I}_A(-t))\mathcal{F}(\mathcal{I}_B(t)), \nonumber\\
    \mathcal{C}_{AB}(\tau) &= \mathcal{F}^{-1}(\mathcal{F}(\mathcal{I}_A(-t))\mathcal{F}(\mathcal{I}_B(t))).
\end{align}
Furthermore, the \textit{timelag} between channels $A$ and $B$ is defined as,
\begin{equation}
    \tau_{AB} = \argmax_{\tau}\,\mathcal{C}_{AB}(\tau).
\end{equation}
\
\begin{figure}
    \plotone{../figures/correlation_1d_example.pdf}
    \caption{Normalized pixel-averaged intensity timeseries for all six AIA EUV channels (top) and cross-correlation curves for selected channel pairs (bottom) for the ``cooling'' case.}
    \label{fig:correlation_1d}
\end{figure}

By exploiting the definition of the cross-correlation as given in \autoref{eq:cc}, we can leverage existing Fourier transform algorithms in order to compute the $\mathcal{C}_{AB}$ in a scalable and efficient manner. For a 500 pixel by 500 pixel active region observation and 15 possible channel pairs, we need to compute $\tau_{AB}$ nearly $4\times10^6$ times. We use the highly-optimized and thoroughly tested Fourier transform algorithms in the NumPy python package for array computations \citep{oliphant_guide_2006} combined with the Dask library for parallel and distributed computing \citep{dask_development_team_dask_2016}. Using Dask, we are able to parallelize this operation over many cores such that, on a 64-core machine, we are able to compute timelags for all 15 channel pairs in every pixel of the \AR in approximately twenty minutes. 

\subsection{Emission Measure Distributions}\label{em_dist}

In addition to the timelag, we also compute the emission measure distribution (\dem) using the method of \citet{hannah_differential_2012} using both our simulated and observed AIA data. \dem provides a measure of how much plasma is emitting over a range of temperatures and has been used by many workers to assess heating frequency in \AR cores \citep[][and references therein]{tripathi_emission_2011,warren_constraints_2011,warren_systematic_2012,schmelz_cold_2012,bradshaw_diagnosing_2012,reep_diagnosing_2013,barnes_inference_2016,barnes_inference_2016,barnes_inference_2016-1}. 

A common technique is to fit a power-law $\mathrm{EM}(T)\sim T^a$ to \dem between approximately 1 and 4 MK and is effectively a measure of how isothermal the distribution is. Many workers \citep[see Table 3 of][and references therein]{bradshaw_diagnosing_2012} have found $2<a<5$. \citet{cargill_active_2014} used a range of waiting times dependent on the event energy (see \autoref{heating}) and found that he could account for the observed range of slopes. Thus, the \dem slope $a$ is one important diagnostic for assessing how often a single loop may be reheated. We compute $a$ between $10^6$ and $10^{6.6}$ K for each pixel in our \AR.
