%%%%%%%%%%%%%%%%%%%%%%%%%%%%%%%%%%%%%%%%%%%%%%%%%%%%%%%%%%%%%%%%%%%%%%%%%%%%%%%
%                                   Results                                   %
%%%%%%%%%%%%%%%%%%%%%%%%%%%%%%%%%%%%%%%%%%%%%%%%%%%%%%%%%%%%%%%%%%%%%%%%%%%%%%%
\section{Results}\label{results}
\authorcomment1{High, intermediate, and low frequency cases plus control models. How should these be divided up?}

Two questions we want to try to answer:
\begin{enumerate}
    \item How is the timelag related to the heating frequency?
    \item What heating model is most consistent with observed timelags?
\end{enumerate}

%%%%%%%%%%%%%%%%%%%%%%%%%%%%%%%%%%%%%%% Intensities %%%%%%%%%%%%%%%%%%%%%%%%%%%
\subsection{Intensities}

We compute the intensities for the 94, 131, 171, 193, 211, and 335 \AA channels of SDO/AIA using the procedure described in \autoref{forward}. We compute the intensity in each pixel of the model \AR over a total simulation period of $3\times10^4$ s$\approx8.3$ hours with the exception of the cooling case which is only run for $10^4$ s. We complete this procedure for each of the five heating scenarios described in \autoref{heating}. 

\autoref{fig:intensity_map} shows a snapshot of the intensity in each pixel of the \AR for each AIA channel at $t=15\times10^3$ for the three nanoflare heating cases.

\begin{figure*}
    \plotone{../figures/intensity_maps.pdf}
    \caption{Snapshots of intensity across the whole \AR at $t=15\times10^3$ s. The rows correspond to the six EUV channels of AIA and the columns are the three different heating frequencies. In each row, the colarbar is on a square root scale and is normalized between zero and the maximum intensity in the low-frequency case. The color tables are the standard AIA color tables as implemented in SunPy \citep{sunpy_community_sunpypython_2015}.}
    \label{fig:intensity_map}
\end{figure*}

%%%%%%%%%%%%%%%%%%%%%%%%%%%%%%%%%%%%%%% Timelags %%%%%%%%%%%%%%%%%%%%%%%%%%%%%%
\subsection{Timelags}

We apply the timelag method of \citet{viall_evidence_2012} to our simulated intensities for all of the heating scenarios discussed in \ref{heating}. The details of the timelag calculation are given in \ref{timelags}. For each pixel in the active region, we compute the cross-correlation for every possible channel pair (15 in total) and find the temporal offset which maximizes the correlation. This offset, or \textit{timelag}, given by \ref{eq:timelag}, indicates how much the second light curve is shifted in time compared to the first. By convention, we order the channel pairs such that the hot channel is listed first, meaning that a positive timelag corresponds to the peak in the cooler channel following the peak in the hotter channel, thus indicating cooling plasma. If the correlation in a given pixel is too low ($\max{\mathcal{C}_{AB}}<0.1$), we mask that pixel and color it white. 

\begin{figure*}[!b]
    \plotone{../figures/model_timelags_1.pdf}
    \caption{Timelag maps for three different channel pairs for all five of the heating models described in \autoref{heating}. The value of each pixel indicates the temporal offset which maximizes the cross-correlation (see \ref{eq:timelag}). The columns indicate the different channel pairs and the rows indicate the three heating scenarios plus our two control cases. The colorbar ranges from -5000 s to +5000 s.}
    \label{fig:model_timelags}
\end{figure*}

\autoref{fig:model_timelags} shows $\tau_{AB}$ in each pixel of our simulated \AR for three channel pairs, 94-335 \AA, 211-131 \AA, and 193-1171 \AA, and for all heating scenarios listed in \autoref{tab:heating}.

\begin{figure*}[!t]
    \plotone{../figures/model_timelags_2.pdf}
    \figurenum{\ref*{fig:model_timelags}}
    \caption{(continued)}
\end{figure*}

\authorcomment1{Compare different heating frequencies for different channel pairs using maps and histograms; show how going from high to low frequencies relaxes to the cooling time between peak temperatures of channels}

\begin{figure*}
    \plotone{../figures/model_timelags_histograms.pdf}
    \caption{Histograms of timelag values across the whole \AR. The rows indicate the different channel pairs and the columns indicate the different heating models. The different colors are used to denote the different heating models. The bin range is $\pm5000$ s and the bin width is 30 s. As with the timelag maps, we do not include timelags with cross-correlation values $<0.1$.}
    \label{fig:timelag_histograms}
\end{figure*}

\autoref{fig:timelag_histograms} shows histograms of timelags for every possible channel pair and all five heating scenarios. Each histogram is colored according the corresponding heating function. The timelags are binned between -5000 s and +5000 s in 30 s bins. The columns are arranged such that heating frequency decreases from left to right while coherence increases from left to right, with the ``cooling only'' case representing maximally coherent evolution over the whole \AR.

Looking at each channel pair, we make two immediate observations: 
\begin{enumerate}
\item As the heating frequency decreases, the number of negative timelags decreases for nearly every channel pair.
\item In over half of the channel pairs, the distribution of positive timelags narrows as the heating frequency decreases
\end{enumerate}

Regarding our first point, in the cooling case, the negative timelag distribution goes to zero for many of the channel pairs, the few exceptions being those pairs involving strongly doubly-peaked channels, i.e. 94 \AA and 131 \AA. In the case of these channels which have both a hot and a cool peak, we expect to find negative timelags, even in our maximally coherent cooling case, because our convention of ordering the ``hot'' channel first has been violated. Thus, cooling plasma can lead to negative timelags. For the remaining channel pairs, negative timelags are associated with the loop heating and cooling cycle being interrupted by repeated events on a given strand. 

Dealing with our second point, nearly all of those pairs that do not exhibit this behavior include either the doubly-peaked 94 \AA and/or 131 \AA channels. 

\authorcomment2{Figure showing timelag maps for all 5 heating models; need to select which channel pairs: 131 pair, 94 pair, what else? Say three channel pairs so we have a 5-by-3 panel figure, split 3-by-3, 2-by-3 across two pages.}

\authorcomment1{Give a thorough qualitative discussion about how the heating frequency is related to the timelags}

\authorcomment1{Relationship to properties of the magnetic field? Energy density?}

%%%%%%%%%%%%%%%%%%%%%%%%%%%%%%%%%%%%%%% Emission Measure Slopes %%%%%%%%%%%%%%%%
\subsection{Emission Measure Slopes}

In addition to the timelag, we also compute the emission measure distribution (\dem) using the method of \citet{hannah_differential_2012} using both our simulated and observed AIA data. \dem provides a measure of how much plasma is emitting over a range of temperatures and has been used by many workers to assess heating frequency in \AR cores \citep[][and references therein]{tripathi_emission_2011,warren_constraints_2011,warren_systematic_2012,schmelz_cold_2012,bradshaw_diagnosing_2012,reep_diagnosing_2013,barnes_inference_2016,barnes_inference_2016,barnes_inference_2016-1}. 

A common technique is to fit a power-law $\mathrm{EM}(T)\sim T^a$ to \dem between approximately 1 and 4 MK and is effectively a measure of how isothermal the distribution is. Many workers \citep[see Table 3 of][and references therein]{bradshaw_diagnosing_2012} have found $2<a<5$. \citet{cargill_active_2014} used a range of waiting times dependent on the event energy (see \autoref{heating}) and found that he could account for the observed range of slopes. Thus, the \dem slope $a$ is one important diagnostic for assessing how often a single loop may be reheated. We compute $a$ between $10^6$ and $10^{6.6}$ K for each pixel in our \AR.

\authorcomment1{Not too sure how this fits in here...just show the modeled slopes? Could also look at relationship between emission measure slope and the timelag}

%%%%%%%%%%%%%%%%%%%%%%%%%%%%%%%%%%%%%%% Observational Comparisons %%%%%%%%%%%%%
\subsection{Comparison with Observations}\label{compare_obs}

\begin{figure*}
    \plotone{../figures/observed_timelags.pdf}
    \caption{Timelag maps as calculated from intensity data from observations of \AR NOAA 1158 by SDO/AIA. Timelag maps are shown for every possible channel pair as indicated in the upper left corner of each map. As in \autoref{fig:model_timelags}, the colorbar range from -5000 s to +5000 s.}
    \label{fig:observed_timelags}
\end{figure*}


\authorcomment1{Show the observed timelags}

\authorcomment1{Emphasize the point that no single heating model is consistent with the observations. Need multiple heating frequencies; this will lead into random forest stuff}

\authorcomment1{Describe random forest technique, results of the classification}

\begin{figure*}
    \plotone{../figures/probability_maps.pdf}
    \caption{Foo bar}
    \label{fig:probability_maps}
\end{figure*}

\begin{figure}
    \plotone{../figures/frequency_map.pdf}
    \caption{foo bar}
    \label{fig:frequency_map}
\end{figure}
