%%%%%%%%%%%%%%%%%%%%%%%%%%%%%%%%%%%%%%%%%%%%%%%%%%%%%%%%%%%%%%%%%%%%%%%%%%%%%%%
%                                   Results                                   %
%%%%%%%%%%%%%%%%%%%%%%%%%%%%%%%%%%%%%%%%%%%%%%%%%%%%%%%%%%%%%%%%%%%%%%%%%%%%%%%
\section{Results}\label{results}
\authorcomment1{High, intermediate, and low frequency cases plus control models. How should these be divided up?}

Two questions we want to try to answer:
\begin{enumerate}
    \item How is the timelag related to the heating frequency?
    \item What heating model is most consistent with observed timelags?
\end{enumerate}

\subsection{Intensities}

\begin{figure*}[!b]
    \plotone{../figures/intensity_maps_1.pdf}
    \caption{Snapshots of intensity across the whole \AR at $t=15\times10^3$ s. The rows correspond to the six EUV channels of AIA and the columns are the three different heating frequencies. In each row, the colarbar is on a square root scale and is normalized between zero and the maximum intensity in the low-frequency case. The color tables are the standard AIA color tables as implemented in SunPy \citep{sunpy_community_sunpypython_2015}.}
    \label{fig:intensity_map}
\end{figure*}
\begin{figure*}[!t]
    \plotone{../figures/intensity_maps_2.pdf}
    \figurenum{\ref*{fig:intensity_map}}
    \caption{(continued)}
\end{figure*}

\authorcomment1{Compare different heating frequencies for different channel pairs using maps and histograms}

\authorcomment1{Compare with observations shown in \citet{viall_survey_2017}.}

\subsection{Timelags}

\subsection{Emission Measure Slopes}

\subsection{Comparison with Observations}\label{compare_obs}

\authorcomment1{Here, show observed timelags (do not really need to show intensities I don't think...) and observed EM slopes}


\authorcomment1{Describe random forest technique, results of the classification}